\documentclass[../../../main.tex]{subfiles}
\begin{document}

%%%%%%%%%%%%%%%%%%%%%%%%%%%%%%%%%%%%%%%%%
%%%%%%%%%%%%%%%%%%%%%%%%%%%%%%%%%%%%%%%%%
%%%%%%%%%%%%%%%%%%%%%%%%%%%%%%%%%%%%%%%%%
\chapter{Further Reading}

To pursue graph theory further, the following list may offer some helpful starting points. 

\begin{itemize}

  \item \citet[ch. 5]{BurgerAndStarbird2010} offers an introductory level discussion of a few different topics in graph theory. This is an excellent place to start reading first.

  \item \citet{Trudeau1993} provides an excellent introduction to graph theory, and to higher mathematics in general. The pace is slow, and the proofs are fully explained. It offers a good entry point to graph theory for the beginner, and going through this text slowly can really pay off.
  
  \item \citet{Chartrand1977} provides an introductory discussion of a number of topics in graph theory.
  
  \item \citet[ch. 8]{Jongsma2019} offers a rigorous introduction to the basic theorems and proofs behind a couple of topics in graph theory.
  
  \item \citet[chs. 6--8]{Flegg1974} provides a more conceptual discussion of some of the classic concepts in graph theory.
  
  \item \citet{BenjaminEtAl2015} presents a variety of classic topics from graph theory. The authors provide a good amount of historical detail, as well as decent discussion of the classic theorems and proofs.

  \item \citet{HartsfieldAndRingel1990} is another thorough introduction to classic topics in graph theory.

  \item \citet{Fournier2009} provides a nice discussion of many applications of graph theory. Its discussion of the \math/ itself is often fairly brief, so it is useful mainly for its discussion of applications. 

  \item \citet{Gould2012} offers a very thorough discussion of the classic topics of graph theory. One of the great features of this book is that it discusses the algorithms behind many of these topics, and so it is very useful for computer scientists.

\end{itemize}

\end{document}
