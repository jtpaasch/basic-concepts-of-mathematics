\documentclass[../../../main.tex]{subfiles}
\begin{document}

%%%%%%%%%%%%%%%%%%%%%%%%%%%%%%%%%%%%%%%%%
%%%%%%%%%%%%%%%%%%%%%%%%%%%%%%%%%%%%%%%%%
%%%%%%%%%%%%%%%%%%%%%%%%%%%%%%%%%%%%%%%%%
\chapter{Listing Infinities}
\label{ch:listing-infinities}

\newtopic{W}{e can compare two sets} for size, without having to count them. As we saw in \chapterref{ch:measuring-size}, if we can construct an \vocab{isomorphism} between the sets $\set{A}$ and $\set{B}$, then it follows that $\set{A}$ and $\set{B}$ are the \vocab{same size}. 

We can use this technique on \vocab{infinite} sets too. If we can construct an isomorphism between two infinite sets $\set{A}$ and $\set{B}$, then we know that $\set{A}$ and $\set{B}$ must be the same size.




%%%%%%%%%%%%%%%%%%%%%%%%%%%%%%%%%%%%%%%%%
%%%%%%%%%%%%%%%%%%%%%%%%%%%%%%%%%%%%%%%%%
\section{A Canonical Infinite Set}

\newthought{When you measure things}, you usually have to pick something as the ``gold standard,'' i.e., the \vocab{canonical} (official) version that you compare everything else to.

In our case, we need to pick some infinite set, and agree that we will treat it as our canonical example. Once we have agreed that this particular set is the ``gold standard'' against which all others will be measured, then we will be able to compare other infinite sets to it.

\begin{terminology}
  We need to elect one particular infinite set to serve as the standard that we will compare other infinite sets against. Here, we will elect the natural numbers for this role: $\Nats/$ will be the \vocab{canonical} example of an infinite set. 
\end{terminology}

Which set should we choose as our canonical example? There are options, but the natural numbers is a natural and simple choice. So, let's take the natural numbers $\Nats/$ as our standard example of an infinite set. Let's agree that $\Nats/$ will be the canonical version of an infinite set as we proceed.

\begin{fdefinition}[The canonical infinite set]
  \label{def:canonical-infinite-set}
  We will say that the natural numbers $\Nats/$ is the \vocab{canonical infinite set}. 
\end{fdefinition}

Now that we have designated $\Nats/$ as our canonical infinite set, let's designate a special symbol to indicate its size. In popular culture, I think a lot of people know this symbol:

\begin{aside}
  \begin{remark}
    The infinity symbol (``$\infty$,'' i.e., a sideways figure eight) is also called a \vocab{lemniscate}, and it is basically a picture of a ribbon that you could go round and round without end. It sometimes has other uses in \math/, but it is not used to indicate the \emph{size} of infinite sets. For that, \mathers/ use the Hebrew letter \vocab{Aleph} ($\aleph$). This symbol was first used by Georg Cantor (1845--1918), who first worked out in rigorous detail the material we are covering here.
  \end{remark}
\end{aside}

\begin{equation*}
  \infty
\end{equation*}

But \mathers/ have chosen a different symbol to denote the size of $\Nats/$. It is this:

\begin{equation*}
  \AlephZero/
\end{equation*}

You can pronounce that out loud as ``Aleph-zero,'' or ``Aleph-null,'' or ``Aleph-nought,'' or something along those lines. 

Instead of $\infty$, let's use Aleph-zero to denote the size of $\Nats/$. Hence, if we want to write that the \emph{size} (i.e., the \vocab{cardinality}) of the natural numbers $\Nats/$ is $\AlephZero/$, we can write this:

\begin{equation*}
  \cardinality{\Nats/} = \AlephZero/
\end{equation*}

Read that out loud like this: ``the cardinality (or size) of the set of natural numbers is Aleph-zero.''

\begin{fdefinition}[The size of $\Nats/$]
  \label{def:aleph-zero}
  We will denote the \vocab{cardinality} (size) of the natural numbers $\Nats/$ with the \vocab{Aleph-zero} symbol, namely $\AlephZero/$. In other words: $\cardinality{\Nats/} = \AlephZero/$.
\end{fdefinition}

\begin{aside}
  \begin{remark}
    Recall from \chapterref{ch:measuring-size} that infinity is not a particular number, i.e., it is not an element in the natural numbers. Hence, when we say that $\AlephZero/$ is not a particular number, we mean that it does not denote any particular element of $\Nats/$. It is just a special symbol to designate the \vocab{size} of $\Nats/$.
  \end{remark}
\end{aside}

This is not a \vocab{particular number}, like ``$10$'' or ``$808,234,093$'' are. Rather, it is a \vocab{size}. It designates the \emph{size} of $\Nats/$. We need a special symbol like this because $\Nats/$ extends outwards, number after number, forever. We can't use a number to name that size, so we have to use a special, invented symbol ($\AlephZero/$).


%%%%%%%%%%%%%%%%%%%%%%%%%%%%%%%%%%%%%%%%%
%%%%%%%%%%%%%%%%%%%%%%%%%%%%%%%%%%%%%%%%%
\section{Comparing Infinities}

\newthought{With $\Nats/$ as our canonical example} of an infinite set, we can now compare other infinite sets to $\Nats/$, to see if those other sets are the same size or not. And as we said, we will do this by constructing an isomorphism between them. If we can build an isomorphism from $\Nats/$ to the others, then we can conclude that they are the same size (they \emph{also} have the size $\AlephZero/$).

\begin{fexample}

As a first example, let's take the set of integers $\Ints/$, from $0$ on up. So, we're taking $0$, and then all of the positive integers, and we're putting those into their own set. Let's call this set $\PositiveIntsAndZero/$. It is this:

\begin{equation*}
  \PositiveIntsAndZero/ = \{ 0, 1, 2, 3, \ldots~ \}
\end{equation*}

Let's compare this to our canonical example $\Nats/$, and let's see if we can build an isomorphism. If we can do that, then we can conclude that $\Nats/$ and $\PositiveIntsAndZero/$ are the same size. 

How can we build an isomorphism from $\Nats/$ to $\PositiveIntsAndZero/$? In this case, it's easy to see that these two sets line up exactly, so we can just map $0 \mapsto 0$, $1 \mapsto 1$, $2 \mapsto 2$ and so on. Let's draw the mapping sideways, like this:

\begin{aside}
  \begin{remark}
    To be more explicit, the function looks like this:
    
    \begin{align*}
      &\func{f}(0) = 0 \hskip 0.5cm
      \func{f}(1) = 1 \hskip 0.5cm
      \func{f}(2) = 2 \hskip 0.5cm
      \\
      &\func{f}(3) = 3 \hskip 0.5cm
      \func{f}(4) = 4 \hskip 0.5cm
      \ldots
    \end{align*}
  \end{remark}
\end{aside}

\begin{center}
  \begin{tabular}{ c c c c c c }
    $\Nats/:$                & $0$   & $1$   & $2$   & $3$   & \ldots \\
                             & $\v/$ & $\v/$ & $\v/$ & $\v/$ &        \\ 
    $\PositiveIntsAndZero/:$ & $0$   & $1$   & $2$   & $3$   & \ldots
  \end{tabular}
\end{center}

This mapping from $\Nats/$ to $\PositiveIntsAndZero/$ is clearly a bijective function, and so it is an \vocab{isomorphism} between these two sets. Hence, we can conclude that $\PositiveIntsAndZero/$ is the \vocab{same size} as $\Nats/$. And since $\Nats/$ has a size of $\AlephZero/$, then we can conclude that $\PositiveIntsAndZero/$ is also that big. It too has a size of $\AlephZero/$. Hence:

\begin{equation*}
  \cardinality{\PositiveIntsAndZero/} = \AlephZero/
\end{equation*}

\end{fexample}

\begin{example}

Now, let's take the zero and all of the \emph{negative} numbers from $\Ints/$, and let's put them into a set all their own. Let's call this set $\NegativeIntsAndZero/$. It is this:

\begin{equation*}
  \NegativeIntsAndZero/ = \{ \ldots, -3, -2, -1, 0 \}
\end{equation*}

Is this the same size as $\Nats/$? If we can construct an isomorphism from $\Nats/$ to $\NegativeIntsAndZero/$, then we can conclude that it is the same size. 

So, can we construct an isomorphism? Yes, by just listing the negative numbers in reverse order. Like this:

\begin{aside}
  \begin{remark}
    To be more explicit, the function looks like this:
    
    \begin{align*}
      \func{f}(0) = 0 \hskip 0.25cm
      &\func{f}(1) = -1 \hskip 0.25cm
      \func{f}(2) = -2 \hskip 0.25cm
      \\
      \func{f}(3) = -3 \hskip 0.25cm
      &\func{f}(4) = -4 \hskip 0.25cm
      \ldots
    \end{align*}
  \end{remark}
\end{aside}

\begin{center}
  \begin{tabular}{ c c c c c c }
    $\Nats/$                & $0$   & $1$   & $2$   & $3$   & \ldots \\
                            & $\v/$ & $\v/$ & $\v/$ & $\v/$ &        \\ 
    $\NegativeIntsAndZero/$ & $0$   & $-1$  & $-2$  & $-3$  & \ldots
  \end{tabular}
\end{center}

Since we can construct an isomorphism here, we can conclude that $\NegativeIntsAndZero/$ is the same size as $\Nats/$, and hence it has the size $\AlephZero/$ too:

\begin{equation*}
  \cardinality{\NegativeIntsAndZero/} = \AlephZero/
\end{equation*}

\end{example}


%%%%%%%%%%%%%%%%%%%%%%%%%%%%%%%%%%%%%%%%%
%%%%%%%%%%%%%%%%%%%%%%%%%%%%%%%%%%%%%%%%%
\section{Listable Sets}

\newthought{There is another way} to think about what we have been doing here. When we construct an isomorphism from $\Nats/$ to some other set $\set{A}$, what exactly are we doing? 

\begin{aside}
  \begin{remark}
    Think of this as an infinitely long shopping list. E.g.,
    
    \begin{diagram}

  \draw (0, -6) -- (0, 0) -- (5.5, 0) -- (5.5, -6);
  \draw (0, -6) -- (-0.25, -6) -- (-0.25, -6.1) -- 
        (2.625, -7.25) -- 
        (5.75, -6.1) -- (5.75, -6) -- (5.5, -6);
  \node at (0.375, -1) [label=right:{0. \fillinblank{3.75cm}}] {};
  \node at (2, -0.85) {$\mathtt{Flour}$};
  \node at (0.375, -2) [label=right:{1. \fillinblank{3.75cm}}] {};
  \node at (2, -1.85) {$\mathtt{Eggs}$};
  \node at (0.375, -3) [label=right:{2. \fillinblank{3.75cm}}] {};
  \node at (2, -2.85) {$\mathtt{Sugar}$};
  \node at (0.375, -4) [label=right:{3. \fillinblank{3.75cm}}] {};
  \node at (2, -3.85) {$\mathtt{Beans}$};
  \node at (2.375, -4.825) [label=right:{$\vdots$}] {};
  
\end{diagram}
  \end{remark}
\end{aside}

Well, one way to look at it is this: we are \vocab{listing} the elements in $\set{A}$, one by one. Imagine that we have an empty shopping list which is infinitely long. Something like this:

\begin{diagram}

  \draw (0, 0) -- (8, 0) -- (8, -6) -- (0, -6) -- (0, 0);
  \node at (0.75, -1) [label=right:{0. \fillinblank{5cm}}] {};
  \node at (0.75, -2) [label=right:{1. \fillinblank{5cm}}] {};
  \node at (0.75, -3) [label=right:{2. \fillinblank{5cm}}] {};
  \node at (0.75, -4) [label=right:{3. \fillinblank{5cm}}] {};
  \node at (0.825, -4.825) [label=right:{$\vdots$}] {};
  
\end{diagram}

To construct an isomorphism from $\Nats/$ to any set $\set{A}$, we basically need to fill in each blank on this list with an item from $\set{A}$. And to do that, we just write in the items from $\set{A}$, one by one.

For instance, with $\NegativeIntsAndZero/$, first we take ``$0$'' and we write it in the first blank:

\begin{diagram}

  \draw (0, 0) -- (8, 0) -- (8, -6) -- (0, -6) -- (0, 0);
  \node at (0.75, -1) [label=right:{0. \fillinblank{5cm}}] {};
  \node at (4.15, -0.85) {$\mathtt{0}$};
  \node at (0.75, -2) [label=right:{1. \fillinblank{5cm}}] {};
  \node at (0.75, -3) [label=right:{2. \fillinblank{5cm}}] {};
  \node at (0.75, -4) [label=right:{3. \fillinblank{5cm}}] {};
  \node at (0.825, -4.825) [label=right:{$\vdots$}] {};
  
\end{diagram}

Then we take ``$-1$'' and we write it in the next blank:

\begin{diagram}

  \draw (0, 0) -- (8, 0) -- (8, -6) -- (0, -6) -- (0, 0);
  \node at (0.75, -1) [label=right:{0. \fillinblank{5cm}}] {};
  \node at (4.15, -0.85) {$\mathtt{0}$};
  \node at (0.75, -2) [label=right:{1. \fillinblank{5cm}}] {};
  \node at (4, -1.85) {$\mathtt{-1}$};
  \node at (0.75, -3) [label=right:{2. \fillinblank{5cm}}] {};
  \node at (0.75, -4) [label=right:{3. \fillinblank{5cm}}] {};
  \node at (0.825, -4.825) [label=right:{$\vdots$}] {};
  
\end{diagram}

Then we write ``$-2$'' in the next blank, then ``$-3$'' in the blank after that, and so on:

\begin{diagram}

  \draw (0, 0) -- (8, 0) -- (8, -6) -- (0, -6) -- (0, 0);
  \node at (0.75, -1) [label=right:{0. \fillinblank{5cm}}] {};
  \node at (4.15, -0.85) {$\mathtt{0}$};
  \node at (0.75, -2) [label=right:{1. \fillinblank{5cm}}] {};
  \node at (4, -1.85) {$\mathtt{-1}$};
  \node at (0.75, -3) [label=right:{2. \fillinblank{5cm}}] {};
  \node at (4, -2.85) {$\mathtt{-2}$};
  \node at (0.75, -4) [label=right:{3. \fillinblank{5cm}}] {};
  \node at (4, -3.85) {$\mathtt{-3}$};
  \node at (0.825, -4.825) [label=right:{$\vdots$}] {};
  
\end{diagram}

\begin{ponder}
  If you like, you can imagine for the sake of the argument that there is something unlimitedly intelligent and powerful (such as the being that many call ``God'') which has the computational capacity to be able to complete the entire list here, all at once. 
\end{ponder}

Of course, for infinite sets, we humans could never complete the process of \vocab{listing out} all of the elements in $\NegativeIntsAndZero/$, since that would go on and on, forever. Nevertheless, we can easily imagine filling it in, endlessly.

So, we can see an isomorphism from $\Nats/$ to another set $\set{A}$ as a \vocab{listing} of all the elements in $\set{A}$. The set $\Nats/$ really just provides the numbering of the blanks that we see going down the left side:

\begin{diagram}

  \draw[color=gray] (0, 0) -- (8, 0) -- (8, -6) -- (0, -6) -- (0, 0);
  \node at (0.75, -1) [label=right:{0. \fillinblank{5cm}}] {};
  \node at (4.15, -0.85) {$\mathtt{0}$};
  \node at (0.75, -2) [label=right:{1. \fillinblank{5cm}}] {};
  \node at (4, -1.85) {$\mathtt{-1}$};
  \node at (0.75, -3) [label=right:{2. \fillinblank{5cm}}] {};
  \node at (4, -2.85) {$\mathtt{-2}$};
  \node at (0.75, -4) [label=right:{3. \fillinblank{5cm}}] {};
  \node at (4, -3.85) {$\mathtt{-3}$};
  \node at (0.825, -4.825) [label=right:{$\vdots$}] {};

  \draw[color=gray,fill=white]
    (-2.5, -2) -- (0.5, -2) -- (0.5, -3.75) -- (-2.5, -3.75) -- (-2.5, -2);
  \node (l1) at (-1, -2.5) {Numbering};
  \node (l1a) at (-1, -3) {from $\Nats/$};
  \draw[->,dashed] (-1, -2) to[out=90,in=180] (0.5, -1);
  \draw[dashed] 
    (0.625, -0.5) -- (1.75, -0.5) -- (1.75, -5.5) -- 
    (0.625, -5.5) -- (0.625, -0.5);

  \draw[color=gray,fill=white]
    (5.5, -2) -- (8.5, -2) -- (8.5, -3.75) -- (5.5, -3.75) -- (5.5, -2);
  \node (l2) at (7, -2.5) {Values};
  \node (l2a) at (7, -3) {from $\NegativeIntsAndZero/$};
  \draw[->,dashed] (7, -2) to[out=90,in=0] (5, -0.75);
  \draw[dashed]
    (3.5, -0.25) -- (4.75, -0.25) -- (4.75, -5) -- (3.5, -5) -- (3.5, -0.25);
  
\end{diagram}

So, let us say that an isomorphism from $\Nats/$ to a set $\set{A}$ is a \vocab{listing} (or synonymously, an \vocab{enumeration}) of $\set{A}$, since the isomorphism ``lists out'' or enumerates all the items in $\set{A}$, by pairing each one up with a number from $\Nats/$. 

\begin{terminology}
  A \vocab{listing} (or \vocab{enumeration}) of an infinite set $\set{A}$ is an isomorphism $\func{f}$ from $\Nats/$ to $\set{A}$. In essence, $\func{f}$ gives us a way to list every element in $\set{A}$.
\end{terminology}

\begin{fdefinition}[Listings]
  \label{def:listings}
  For any infinite set $\set{A}$, we will say that an isomorphism $\funcsig{f}{\Nats/}{\set{A}}$ is a \vocab{listing} (or synonymously, an \vocab{enumeration}) of $\set{A}$.
\end{fdefinition}

If we can construct a listing of an infinite set $\set{A}$, then let us say that it is \vocab{listable} (or synonymously, \vocab{enumerable}). So, if we have an infinite set $\set{A}$, and we are able to construct a listing of it (i.e., an isomorphism of it), then we can say that it is listable. It's items can be listed or enumerated on this infinitely long shopping list that we have been talking about.

\begin{terminology}
  A set $\set{A}$ is \vocab{listable} (or \vocab{enumerable}) if a listing (enumeration) can be constructed. If you prefer to imagine a being with unlimited computational power (e.g., such as the being many call ``God''), then you might imagine that a set is listable if it is the sort of thing that God could list out or enumerate on an infinitely long shopping list.
\end{terminology}

\begin{fdefinition}[Listability]
  \label{def:listability}
  For any infinite set $\set{A}$, we will say that $\set{A}$ is \vocab{listable} (or synonymously, \vocab{enumerable}) if a listing (enumeration) can be constructed.

\end{fdefinition}

Why are we setting down these definitions? The reason is this. It turns out that not every infinite set is listable. There are certain infinite sets for which no listing cannot be constructed. That is a topic we will turn to soon.


%%%%%%%%%%%%%%%%%%%%%%%%%%%%%%%%%%%%%%%%%
%%%%%%%%%%%%%%%%%%%%%%%%%%%%%%%%%%%%%%%%%
\section{Summary}

\newthought{In this chapter}, we looked at listing out or enumerating the items in a set. 

\begin{itemize}

  \item We choose the \vocab{natural numbers} $\Nats/$ as the \vocab{canonical} example of an infinite set. We designate a special symbol for the \vocab{size} of this set: $\AlephZero/$ (pronounced ``Aleph-zero,'' or ``Aleph-null,'' or ``Aleph-nought''). 

  \item A \vocab{listing} (or synonymously, an \vocab{enumeration}) of a set $\set{A}$ is an \vocab{isomorphism} $\func{f}$ from $\Nats/$ to $\set{A}$.
  
  \item A set is \vocab{listable} (or synonymously, \vocab{enumerable}) if a \vocab{listing} of it can be constructed.

\end{itemize}

\end{document}
