\documentclass[../../../main.tex]{subfiles}
\begin{document}

%%%%%%%%%%%%%%%%%%%%%%%%%%%%%%%%%%%%%%%%%
%%%%%%%%%%%%%%%%%%%%%%%%%%%%%%%%%%%%%%%%%
%%%%%%%%%%%%%%%%%%%%%%%%%%%%%%%%%%%%%%%%%
\chapter{Further Reading}

To pursue questions about infinities further, the following list may offer some helpful starting points.

\begin{itemize}

  \item \citet[chs.~1--2]{BoolosBurgessAndJeffrey2002} offers a slow-paced introduction to the ideas of listability (enumerability) and the diagonalization trick.

  \item \citet[ch.~9]{Stewart1995} provides a good but introductory-level discussion of some of the basic concepts that underly our modern discussions of infinity.
  
  \item \citet[ch.~14]{StewartAndTall2015} offer a slightly more advanced introduction to the mathematics of infinities.
  
  \item \citet[ch.~9]{Steinhart2018} provides a non-technical discussion not only to infinities of size $\AlephZero/$, but also to the infinities that are bigger than $\AlephZero/$.
  
  \item \citet[ch.~4]{Wilder2012} provides another non-technical discussion of questions around infinities.
  
  \item \citet[ch.~4]{Zach2019} provides a helpful, somewhat technical discussion of the topics of infinity, cardinality, and diagonalization.
  
  \item \citet[ch.~2]{Cummings2018} offers a slow-paced discussion of cardinality and infinities, along with how to write some of the relevant proofs.
  
  \item \cite[ch.~7]{Pinter2014} provides a simpler textbook-level discussion of infinities and listability, along with proofs of most of the basic facts.
  
  \item \cite[ch.~6]{Enderton1977} offers a textbook-level discussion of infinities, especially in the first couple sections of the chapter. Enderton also offers a different way to traverse the rational numbers so as to list them.
  
\end{itemize}

\end{document}