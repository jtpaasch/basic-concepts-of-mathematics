\documentclass[../../../main.tex]{subfiles}
\begin{document}

%%%%%%%%%%%%%%%%%%%%%%%%%%%%%%%%%%%%%%%%%
%%%%%%%%%%%%%%%%%%%%%%%%%%%%%%%%%%%%%%%%%
%%%%%%%%%%%%%%%%%%%%%%%%%%%%%%%%%%%%%%%%%
\chapter{The Parts of the CPU}
\label{ch:the-parts-of-the-CPU}

\newtopic{I}{n this chapter}, we will look at the basic parts that go into the CPU (Central Processing Unit) of a computer, so that we can understand how it actually works. To keep things simple, we will put together a very basic CPU, but all modern computers are essentially just beefed up versions of the same thing.


%%%%%%%%%%%%%%%%%%%%%%%%%%%%%%%%%%%%%%%%%
%%%%%%%%%%%%%%%%%%%%%%%%%%%%%%%%%%%%%%%%%
\section{Numbers ``in'' the Computer}

\newthought{Previously, we looked at a variety} of ways to encode numbers. For instance, we looked at the decimal system (which is what most of us use daily), we looked at the hexadecimal system, and we looked at the binary system. 

How might we realize any of these number systems inside a computer? How might we ``write down'' a number, inside a computer, so that a computer can do something with that number (like add it to another number)? There isn't a piece of paper inside the computer, so we can't write it down directly. 

But there are wires inside the computer, and we can use wires to represent a number. A wire can have some amount of current flowing through it. So, we can say that if a wire has no current flowing through it, then it represents a ``0,'' and if it has current flowing through it, then it represents a ``1.'' 

So, we have a way to represent a binary digit! Let's imagine that we make a little box, that has a wire running through it, like this:

\begin{diagram}

  \draw (0.25, 0) -- (0.25, 0.75) -- (0.5, 0.75) -- (0.5, 0);
  \draw[color=gray] (0.25, 1.25) -- (0.25, 0.75) -- (0.5, 0.75) -- (0.5, 1.25);
  \draw[color=gray] (0.25, -0.5) -- (0.25, 0) -- (0.5, 0) -- (0.5, -0.5);
  \draw (0, 0) -- (0.75, 0) -- (0.75, 0.75) -- (0, 0.75) -- (0, 0);
  
  \draw[->] (0.375, 1) -- (0.375, 1.5);
  \draw[->] (0.375, -1) -- (0.375, -0.25);
  \node at (0.375, -1.25) {a wire};
  \node at (0.375, -1.75) {we can send};
  \node at (0.375, -2.25) {current through};

  \draw[<-,color=gray] (1, 0.5) -- (1.75, 0.5);
  \node at (3.5, 0.5) {A ``slot'' to hold};
  \node at (3.5, 0) {one binary digit};

\end{diagram}

To represent that current is passing through the wire, let's draw the wire shaded, like this:

\begin{diagram}

  \draw[fill=black] (0.25, 0) -- (0.25, 0.75) -- (0.5, 0.75) -- (0.5, 0);
  \draw[color=gray,fill=gray] (0.25, 1.25) -- (0.25, 0.75) -- (0.5, 0.75) -- (0.5, 1.25);
  \draw[color=gray,fill=gray] (0.25, -0.5) -- (0.25, 0) -- (0.5, 0) -- (0.5, -0.5);
  \draw (0, 0) -- (0.75, 0) -- (0.75, 0.75) -- (0, 0.75) -- (0, 0);
  
  \draw[<-,color=gray] (1, 0.5) -- (1.5, 0.5);
  \node at (4, 0.5) {Current is on the wire,};
  \node at (4, 0) {so this is a ``1''};

\end{diagram}

To represent that there is no current passing through the wire, let's draw the wire with no shading, like this:

\begin{diagram}
  
  \draw (0.25, 0) -- (0.25, 0.75) -- (0.5, 0.75) -- (0.5, 0);
  \draw[color=gray] (0.25, 1.25) -- (0.25, 0.75) -- (0.5, 0.75) -- (0.5, 1.25);
  \draw[color=gray] (0.25, -0.5) -- (0.25, 0) -- (0.5, 0) -- (0.5, -0.5);
  \draw (0, 0) -- (0.75, 0) -- (0.75, 0.75) -- (0, 0.75) -- (0, 0);
  
  \draw[<-,color=gray] (1, 0.5) -- (1.5, 0.5);
  \node at (4, 0.5) {No current on the wire,};
  \node at (4, 0) {so this is a ``0''};

\end{diagram}

So, what we have here is a little component that can represent a binary digit, and it uses nothing but electronics. Next, let's imagine putting together an array of these little components, so we can have more than one digit. Let's imagine four components, side by side:

\begin{diagram}

  \draw (0.25, 0) -- (0.25, 0.75) -- (0.5, 0.75) -- (0.5, 0);
  \draw[color=gray] (0.25, 1.25) -- (0.25, 0.75) -- (0.5, 0.75) -- (0.5, 1.25);
  \draw[color=gray] (0.25, -0.5) -- (0.25, 0) -- (0.5, 0) -- (0.5, -0.5);
  \draw (0, 0) -- (0.75, 0) -- (0.75, 0.75) -- (0, 0.75) -- (0, 0);

  \draw (1.25, 0) -- (1.25, 0.75) -- (1.5, 0.75) -- (1.5, 0);
  \draw[color=gray] (1.25, 1.25) -- (1.25, 0.75) -- (1.5, 0.75) -- (1.5, 1.25);
  \draw[color=gray] (1.25, -0.5) -- (1.25, 0) -- (1.5, 0) -- (1.5, -0.5);
  \draw (1, 0) -- (1.75, 0) -- (1.75, 0.75) -- (1, 0.75) -- (1, 0);

  \draw (2.25, 0) -- (2.25, 0.75) -- (2.5, 0.75) -- (2.5, 0);
  \draw[color=gray] (2.25, 1.25) -- (2.25, 0.75) -- (2.5, 0.75) -- (2.5, 1.25);
  \draw[color=gray] (2.25, -0.5) -- (2.25, 0) -- (2.5, 0) -- (2.5, -0.5);
  \draw (2, 0) -- (2.75, 0) -- (2.75, 0.75) -- (2, 0.75) -- (2, 0);
  
  \draw (3.25, 0) -- (3.25, 0.75) -- (3.5, 0.75) -- (3.5, 0);
  \draw[color=gray] (3.25, 1.25) -- (3.25, 0.75) -- (3.5, 0.75) -- (3.5, 1.25);
  \draw[color=gray] (3.25, -0.5) -- (3.25, 0) -- (3.5, 0) -- (3.5, -0.5);
  \draw (3, 0) -- (3.75, 0) -- (3.75, 0.75) -- (3, 0.75) -- (3, 0);    

\end{diagram}

Each of these components represents one digit, which we call a ``bit'' of information. Since we have four components here, this can represent a 4-bit number. For instance, it represents the number zero (in binary: $0000$) when all of the wires have no current running through them: 

\begin{diagram}

  \draw (0.25, 0) -- (0.25, 0.75) -- (0.5, 0.75) -- (0.5, 0);
  \draw[color=gray] (0.25, 1.25) -- (0.25, 0.75) -- (0.5, 0.75) -- (0.5, 1.25);
  \draw[color=gray] (0.25, -0.5) -- (0.25, 0) -- (0.5, 0) -- (0.5, -0.5);
  \draw (0, 0) -- (0.75, 0) -- (0.75, 0.75) -- (0, 0.75) -- (0, 0);

  \draw (1.25, 0) -- (1.25, 0.75) -- (1.5, 0.75) -- (1.5, 0);
  \draw[color=gray] (1.25, 1.25) -- (1.25, 0.75) -- (1.5, 0.75) -- (1.5, 1.25);
  \draw[color=gray] (1.25, -0.5) -- (1.25, 0) -- (1.5, 0) -- (1.5, -0.5);
  \draw (1, 0) -- (1.75, 0) -- (1.75, 0.75) -- (1, 0.75) -- (1, 0);

  \draw (2.25, 0) -- (2.25, 0.75) -- (2.5, 0.75) -- (2.5, 0);
  \draw[color=gray] (2.25, 1.25) -- (2.25, 0.75) -- (2.5, 0.75) -- (2.5, 1.25);
  \draw[color=gray] (2.25, -0.5) -- (2.25, 0) -- (2.5, 0) -- (2.5, -0.5);
  \draw (2, 0) -- (2.75, 0) -- (2.75, 0.75) -- (2, 0.75) -- (2, 0);
  
  \draw (3.25, 0) -- (3.25, 0.75) -- (3.5, 0.75) -- (3.5, 0);
  \draw[color=gray] (3.25, 1.25) -- (3.25, 0.75) -- (3.5, 0.75) -- (3.5, 1.25);
  \draw[color=gray] (3.25, -0.5) -- (3.25, 0) -- (3.5, 0) -- (3.5, -0.5);
  \draw (3, 0) -- (3.75, 0) -- (3.75, 0.75) -- (3, 0.75) -- (3, 0);    

  \draw[->] (0.375, -1) -- (0.375, -0.25);
  \node at (0.375, -1.5) {$0$};

  \draw[->] (1.375, -1) -- (1.375, -0.25);
  \node at (1.375, -1.5) {$0$};
  
  \draw[->] (2.375, -1) -- (2.375, -0.25);
  \node at (2.375, -1.5) {$0$};
  
  \draw[->] (3.375, -1) -- (3.375, -0.25);
  \node at (3.375, -1.5) {$0$};

\end{diagram}

It represents the number one (in binary: $0001$) when there is current going through the wire on the far right:

\begin{diagram}

  \draw (0.25, 0) -- (0.25, 0.75) -- (0.5, 0.75) -- (0.5, 0);
  \draw[color=gray] (0.25, 1.25) -- (0.25, 0.75) -- (0.5, 0.75) -- (0.5, 1.25);
  \draw[color=gray] (0.25, -0.5) -- (0.25, 0) -- (0.5, 0) -- (0.5, -0.5);
  \draw (0, 0) -- (0.75, 0) -- (0.75, 0.75) -- (0, 0.75) -- (0, 0);

  \draw (1.25, 0) -- (1.25, 0.75) -- (1.5, 0.75) -- (1.5, 0);
  \draw[color=gray] (1.25, 1.25) -- (1.25, 0.75) -- (1.5, 0.75) -- (1.5, 1.25);
  \draw[color=gray] (1.25, -0.5) -- (1.25, 0) -- (1.5, 0) -- (1.5, -0.5);
  \draw (1, 0) -- (1.75, 0) -- (1.75, 0.75) -- (1, 0.75) -- (1, 0);

  \draw (2.25, 0) -- (2.25, 0.75) -- (2.5, 0.75) -- (2.5, 0);
  \draw[color=gray] (2.25, 1.25) -- (2.25, 0.75) -- (2.5, 0.75) -- (2.5, 1.25);
  \draw[color=gray] (2.25, -0.5) -- (2.25, 0) -- (2.5, 0) -- (2.5, -0.5);
  \draw (2, 0) -- (2.75, 0) -- (2.75, 0.75) -- (2, 0.75) -- (2, 0);
  
  \draw[fill=black] (3.25, 0) -- (3.25, 0.75) -- (3.5, 0.75) -- (3.5, 0);
  \draw[color=gray,fill=gray] (3.25, 1.25) -- (3.25, 0.75) -- (3.5, 0.75) -- (3.5, 1.25);
  \draw[color=gray,fill=gray] (3.25, -0.5) -- (3.25, 0) -- (3.5, 0) -- (3.5, -0.5);
  \draw (3, 0) -- (3.75, 0) -- (3.75, 0.75) -- (3, 0.75) -- (3, 0);    

  \draw[->] (0.375, -1) -- (0.375, -0.25);
  \node at (0.375, -1.5) {$0$};

  \draw[->] (1.375, -1) -- (1.375, -0.25);
  \node at (1.375, -1.5) {$0$};
  
  \draw[->] (2.375, -1) -- (2.375, -0.25);
  \node at (2.375, -1.5) {$0$};
  
  \draw[->] (3.375, -1) -- (3.375, -0.25);
  \node at (3.375, -1.5) {$1$};

\end{diagram}

It represents the number 5 (in binary: $0101$) when the second and fourth wire has current going through it:

\begin{diagram}

  \draw (0.25, 0) -- (0.25, 0.75) -- (0.5, 0.75) -- (0.5, 0);
  \draw[color=gray] (0.25, 1.25) -- (0.25, 0.75) -- (0.5, 0.75) -- (0.5, 1.25);
  \draw[color=gray] (0.25, -0.5) -- (0.25, 0) -- (0.5, 0) -- (0.5, -0.5);
  \draw (0, 0) -- (0.75, 0) -- (0.75, 0.75) -- (0, 0.75) -- (0, 0);

  \draw[fill=black] (1.25, 0) -- (1.25, 0.75) -- (1.5, 0.75) -- (1.5, 0);
  \draw[color=gray,fill=gray] (1.25, 1.25) -- (1.25, 0.75) -- (1.5, 0.75) -- (1.5, 1.25);
  \draw[color=gray,fill=gray] (1.25, -0.5) -- (1.25, 0) -- (1.5, 0) -- (1.5, -0.5);
  \draw (1, 0) -- (1.75, 0) -- (1.75, 0.75) -- (1, 0.75) -- (1, 0);

  \draw (2.25, 0) -- (2.25, 0.75) -- (2.5, 0.75) -- (2.5, 0);
  \draw[color=gray] (2.25, 1.25) -- (2.25, 0.75) -- (2.5, 0.75) -- (2.5, 1.25);
  \draw[color=gray] (2.25, -0.5) -- (2.25, 0) -- (2.5, 0) -- (2.5, -0.5);
  \draw (2, 0) -- (2.75, 0) -- (2.75, 0.75) -- (2, 0.75) -- (2, 0);
  
  \draw[fill=black] (3.25, 0) -- (3.25, 0.75) -- (3.5, 0.75) -- (3.5, 0);
  \draw[color=gray,fill=gray] (3.25, 1.25) -- (3.25, 0.75) -- (3.5, 0.75) -- (3.5, 1.25);
  \draw[color=gray,fill=gray] (3.25, -0.5) -- (3.25, 0) -- (3.5, 0) -- (3.5, -0.5);
  \draw (3, 0) -- (3.75, 0) -- (3.75, 0.75) -- (3, 0.75) -- (3, 0);    

  \draw[->] (0.375, -1) -- (0.375, -0.25);
  \node at (0.375, -1.5) {$0$};

  \draw[->] (1.375, -1) -- (1.375, -0.25);
  \node at (1.375, -1.5) {$1$};
  
  \draw[->] (2.375, -1) -- (2.375, -0.25);
  \node at (2.375, -1.5) {$0$};
  
  \draw[->] (3.375, -1) -- (3.375, -0.25);
  \node at (3.375, -1.5) {$1$};

\end{diagram}

So this set of four components can represent any 4-digit number, in binary.


%%%%%%%%%%%%%%%%%%%%%%%%%%%%%%%%%%%%%%%%%
%%%%%%%%%%%%%%%%%%%%%%%%%%%%%%%%%%%%%%%%%
\section{Scratch registers}

\newthought{The component} we built in the last section can hold any 4-bit number, where the number is encoded as a binary number. 

We can build a number of these little components, and put them inside of a computer. That way, when the computer needs to do some scratch work (calculations with numbers), it can use our little components as places where it can ``write down'' the numbers it needs to work with.

We call a component that can hold a number for a computer's scratch work a \vocab{scratch register}. The scratch register we are imagining here is a \vocab{4-bit} scratch register, because it can hold a 4-bit number. 

\begin{aside}
  \begin{remark}
    Most computers today have 32-bit or 64-bit scratch registers. In other words, each scratch register has 32 or 64 ``slots,'' each capable of holding a ``zero'' or a ``one.'' For the sake of simplicity, we will stick to 4-bit registers, but the only difference between a 4-bit register and a 32- or 64-bit register is that the bigger register can hold bigger numbers, since it has more ``slots'' to write down digits in. 
  \end{remark}
\end{aside}

Every modern computer comes with at least a few dozen scratch registers. Each scratch register has a special name too. Let's imagine that we have four scratch registers, named $R0$ (for ``Register 0''), $R1$ (for ``Register 1''), and so on. Like this:

\begin{diagram}

  \node at (-7.5, 3.75) {\textsf{R0}};
  \draw (-7, 3.5) -- (-8, 3.5) -- (-8, 4) -- (-7, 4);
  \draw[color=gray]
    (-6.2, 3.5) -- (-6.2, 4) -- (-6.3, 4) -- (-6.3, 3.5) -- (-6.2, 3.5);
  \draw[color=gray]
    (-6.7, 3.5) -- (-6.7, 4) -- (-6.8, 4) -- (-6.8, 3.5) -- (-6.7, 3.5);
  \draw[color=gray]
    (-5.2, 3.5) -- (-5.2, 4) -- (-5.3, 4) -- (-5.3, 3.5) -- (-5.2, 3.5);
  \draw[color=gray]
    (-5.7, 3.5) -- (-5.7, 4) -- (-5.8, 4) -- (-5.8, 3.5) -- (-5.7, 3.5);
  \draw (-7, 3.5) -- (-6.5, 3.5) -- (-6.5, 4) -- (-7, 4) -- (-7, 3.5);
  \draw (-6.5, 3.5) -- (-6, 3.5) -- (-6, 4) -- (-6.5, 4);
  \draw (-6, 3.5) -- (-5.5, 3.5) -- (-5.5, 4) -- (-6, 4);
  \draw (-5.5, 3.5) -- (-5, 3.5) -- (-5, 4) -- (-5.5, 4);

  \node at (-4.25, 3.75) {\textsf{R1}};
  \draw (-3.75, 3.5) -- (-4.75, 3.5) -- (-4.75, 4) -- (-3.75, 4);
  \draw[color=gray]
    (-2.95, 3.5) -- (-2.95, 4) -- (-3.05, 4) -- (-3.05, 3.5) -- (-2.95, 3.5);
  \draw[color=gray]
    (-3.45, 3.5) -- (-3.45, 4) -- (-3.55, 4) -- (-3.55, 3.5) -- (-3.45, 3.5);
  \draw[color=gray]
    (-1.95, 3.5) -- (-1.95, 4) -- (-2.05, 4) -- (-2.05, 3.5) -- (-1.95, 3.5);
  \draw[color=gray]
    (-2.45, 3.5) -- (-2.45, 4) -- (-2.55, 4) -- (-2.55, 3.5) -- (-2.45, 3.5);
  \draw (-3.75, 3.5) -- (-3.25, 3.5) -- (-3.25, 4) -- (-3.75, 4) -- (-3.75, 3.5);
  \draw (-3.25, 3.5) -- (-2.75, 3.5) -- (-2.75, 4) -- (-3.25, 4);
  \draw (-2.75, 3.5) -- (-2.25, 3.5) -- (-2.25, 4) -- (-2.75, 4);
  \draw (-2.25, 3.5) -- (-1.75, 3.5) -- (-1.75, 4) -- (-2.25, 4);

  \node at (-7.5, 2.75) {\textsf{R2}};
  \draw (-7, 2.5) -- (-8, 2.5) -- (-8, 3) -- (-7, 3);
  \draw[color=gray]
    (-6.2, 2.5) -- (-6.2, 3) -- (-6.3, 3) -- (-6.3, 2.5) -- (-6.2, 2.5);
  \draw[color=gray]
    (-6.7, 2.5) -- (-6.7, 3) -- (-6.8, 3) -- (-6.8, 2.5) -- (-6.7, 2.5);
  \draw[color=gray]
    (-5.2, 2.5) -- (-5.2, 3) -- (-5.3, 3) -- (-5.3, 2.5) -- (-5.2, 2.5);
  \draw[color=gray]
    (-5.7, 2.5) -- (-5.7, 3) -- (-5.8, 3) -- (-5.8, 2.5) -- (-5.7, 2.5);
  \draw (-7, 2.5) -- (-6.5, 2.5) -- (-6.5, 3) -- (-7, 3) -- (-7, 2.5);
  \draw (-6.5, 2.5) -- (-6, 2.5) -- (-6, 3) -- (-6.5, 3);
  \draw (-6, 2.5) -- (-5.5, 2.5) -- (-5.5, 3) -- (-6, 3);
  \draw (-5.5, 2.5) -- (-5, 2.5) -- (-5, 3) -- (-5.5, 3);

  \node at (-4.25, 2.75) {\textsf{R3}};
  \draw (-3.75, 2.5) -- (-4.75, 2.5) -- (-4.75, 3) -- (-3.75, 3);
  \draw[color=gray]
    (-2.95, 2.5) -- (-2.95, 3) -- (-3.05, 3) -- (-3.05, 2.5) -- (-2.95, 2.5);
  \draw[color=gray]
    (-3.45, 2.5) -- (-3.45, 3) -- (-3.55, 3) -- (-3.55, 2.5) -- (-3.45, 2.5);
  \draw[color=gray]
    (-1.95, 2.5) -- (-1.95, 3) -- (-2.05, 3) -- (-2.05, 2.5) -- (-1.95, 2.5);
  \draw[color=gray]
    (-2.45, 2.5) -- (-2.45, 3) -- (-2.55, 3) -- (-2.55, 2.5) -- (-2.45, 2.5);
  \draw (-3.75, 2.5) -- (-3.25, 2.5) -- (-3.25, 3) -- (-3.75, 3) -- (-3.75, 2.5);
  \draw (-3.25, 2.5) -- (-2.75, 2.5) -- (-2.75, 3) -- (-3.25, 3);
  \draw (-2.75, 2.5) -- (-2.25, 2.5) -- (-2.25, 3) -- (-2.75, 3);
  \draw (-2.25, 2.5) -- (-1.75, 2.5) -- (-1.75, 3) -- (-2.25, 3);

\end{diagram}

For example, the computer can ``write down'' the number one (in binary: $0001$) in scratch register $R0$, and it can ``write down'' the number five (in binary: $0101$) in scratch register $R1$:

\begin{diagram}

  \node at (-7.5, 3.75) {\textsf{R0}};
  \draw (-7, 3.5) -- (-8, 3.5) -- (-8, 4) -- (-7, 4);
  \draw[color=gray]
    (-6.2, 3.5) -- (-6.2, 4) -- (-6.3, 4) -- (-6.3, 3.5) -- (-6.2, 3.5);
  \draw[color=gray]
    (-6.7, 3.5) -- (-6.7, 4) -- (-6.8, 4) -- (-6.8, 3.5) -- (-6.7, 3.5);
  \draw[color=gray]
    (-5.7, 3.5) -- (-5.7, 4) -- (-5.8, 4) -- (-5.8, 3.5) -- (-5.7, 3.5);
  \draw[color=gray,fill=black]
    (-5.2, 3.5) -- (-5.2, 4) -- (-5.3, 4) -- (-5.3, 3.5) -- (-5.2, 3.5);    
  \draw (-7, 3.5) -- (-6.5, 3.5) -- (-6.5, 4) -- (-7, 4) -- (-7, 3.5);
  \draw (-6.5, 3.5) -- (-6, 3.5) -- (-6, 4) -- (-6.5, 4);
  \draw (-6, 3.5) -- (-5.5, 3.5) -- (-5.5, 4) -- (-6, 4);
  \draw (-5.5, 3.5) -- (-5, 3.5) -- (-5, 4) -- (-5.5, 4);

  \node at (-9.5, 3.75) {$0001$};
  \draw[->] (-9, 3.75) -- (-8.25, 3.75);

  \node at (-4.25, 3.75) {\textsf{R1}};
  \draw (-3.75, 3.5) -- (-4.75, 3.5) -- (-4.75, 4) -- (-3.75, 4);
  \draw[color=gray]
    (-3.45, 3.5) -- (-3.45, 4) -- (-3.55, 4) -- (-3.55, 3.5) -- (-3.45, 3.5);
  \draw[color=gray,fill=black]
    (-2.95, 3.5) -- (-2.95, 4) -- (-3.05, 4) -- (-3.05, 3.5) -- (-2.95, 3.5);
  \draw[color=gray]
    (-2.45, 3.5) -- (-2.45, 4) -- (-2.55, 4) -- (-2.55, 3.5) -- (-2.45, 3.5);
  \draw[color=gray,fill=black]
    (-1.95, 3.5) -- (-1.95, 4) -- (-2.05, 4) -- (-2.05, 3.5) -- (-1.95, 3.5);
  \draw (-3.75, 3.5) -- (-3.25, 3.5) -- (-3.25, 4) -- (-3.75, 4) -- (-3.75, 3.5);
  \draw (-3.25, 3.5) -- (-2.75, 3.5) -- (-2.75, 4) -- (-3.25, 4);
  \draw (-2.75, 3.5) -- (-2.25, 3.5) -- (-2.25, 4) -- (-2.75, 4);
  \draw (-2.25, 3.5) -- (-1.75, 3.5) -- (-1.75, 4) -- (-2.25, 4);

  \node at (-0.15, 3.75) {$0101$};
  \draw[<-] (-1.5, 3.75) -- (-0.75, 3.75);

  \node at (-7.5, 2.75) {\textsf{R2}};
  \draw (-7, 2.5) -- (-8, 2.5) -- (-8, 3) -- (-7, 3);
  \draw[color=gray]
    (-6.2, 2.5) -- (-6.2, 3) -- (-6.3, 3) -- (-6.3, 2.5) -- (-6.2, 2.5);
  \draw[color=gray]
    (-6.7, 2.5) -- (-6.7, 3) -- (-6.8, 3) -- (-6.8, 2.5) -- (-6.7, 2.5);
  \draw[color=gray]
    (-5.2, 2.5) -- (-5.2, 3) -- (-5.3, 3) -- (-5.3, 2.5) -- (-5.2, 2.5);
  \draw[color=gray]
    (-5.7, 2.5) -- (-5.7, 3) -- (-5.8, 3) -- (-5.8, 2.5) -- (-5.7, 2.5);
  \draw (-7, 2.5) -- (-6.5, 2.5) -- (-6.5, 3) -- (-7, 3) -- (-7, 2.5);
  \draw (-6.5, 2.5) -- (-6, 2.5) -- (-6, 3) -- (-6.5, 3);
  \draw (-6, 2.5) -- (-5.5, 2.5) -- (-5.5, 3) -- (-6, 3);
  \draw (-5.5, 2.5) -- (-5, 2.5) -- (-5, 3) -- (-5.5, 3);

  \node at (-4.25, 2.75) {\textsf{R3}};
  \draw (-3.75, 2.5) -- (-4.75, 2.5) -- (-4.75, 3) -- (-3.75, 3);
  \draw[color=gray]
    (-2.95, 2.5) -- (-2.95, 3) -- (-3.05, 3) -- (-3.05, 2.5) -- (-2.95, 2.5);
  \draw[color=gray]
    (-3.45, 2.5) -- (-3.45, 3) -- (-3.55, 3) -- (-3.55, 2.5) -- (-3.45, 2.5);
  \draw[color=gray]
    (-1.95, 2.5) -- (-1.95, 3) -- (-2.05, 3) -- (-2.05, 2.5) -- (-1.95, 2.5);
  \draw[color=gray]
    (-2.45, 2.5) -- (-2.45, 3) -- (-2.55, 3) -- (-2.55, 2.5) -- (-2.45, 2.5);
  \draw (-3.75, 2.5) -- (-3.25, 2.5) -- (-3.25, 3) -- (-3.75, 3) -- (-3.75, 2.5);
  \draw (-3.25, 2.5) -- (-2.75, 2.5) -- (-2.75, 3) -- (-3.25, 3);
  \draw (-2.75, 2.5) -- (-2.25, 2.5) -- (-2.25, 3) -- (-2.75, 3);
  \draw (-2.25, 2.5) -- (-1.75, 2.5) -- (-1.75, 3) -- (-2.25, 3);

\end{diagram}



%%%%%%%%%%%%%%%%%%%%%%%%%%%%%%%%%%%%%%%%%
%%%%%%%%%%%%%%%%%%%%%%%%%%%%%%%%%%%%%%%%%
\section{Memory}

\newthought{If our computer needs more places} to write things down (when the scratch registers aren't enough), let's also imagine that it will have a stack of extra registers over on the side. We call this stack of side registers the \vocab{memory} of the computer (or to be more exact, its \vocab{Random Access Memory}, also known as \vocab{RAM}).

Let's suppose that these side registers are each 8-bits in size. That is, each one can hold 8 binary digits. So, whereas our scratch registers were 4-bit registers (because they could hold 4 binary digits), these registers are 8-bits in size (because they can hold 8 binary digits). The size ``8-bits'' is used so often we have a special name for it: we call it a \vocab{byte}. So each of these side registers can hold one \emph{byte} of digits. To draw a picture, one of these side registers would look like this:

\begin{diagram}

  \draw[color=gray] (0.2, 0) -- (0.2, 0.5) -- (0.3, 0.5) -- (0.3, 0) -- (0.2, 0);
  \draw[color=gray] (0.7, 0) -- (0.7, 0.5) -- (0.8, 0.5) -- (0.8, 0) -- (0.7, 0);
  \draw[color=gray] (1.2, 0) -- (1.2, 0.5) -- (1.3, 0.5) -- (1.3, 0) -- (1.2, 0);
  \draw[color=gray] (1.7, 0) -- (1.7, 0.5) -- (1.8, 0.5) -- (1.8, 0) -- (1.7, 0);
  \draw[color=gray] (2.2, 0) -- (2.2, 0.5) -- (2.3, 0.5) -- (2.3, 0) -- (2.2, 0);
  \draw[color=gray] (2.7, 0) -- (2.7, 0.5) -- (2.8, 0.5) -- (2.8, 0) -- (2.7, 0);
  \draw[color=gray] (3.2, 0) -- (3.2, 0.5) -- (3.3, 0.5) -- (3.3, 0) -- (3.2, 0);
  \draw[color=gray] (3.7, 0) -- (3.7, 0.5) -- (3.8, 0.5) -- (3.8, 0) -- (3.7, 0);
  \draw (0, 0) -- (0.5, 0) -- (0.5, 0.5) -- (0, 0.5) -- (0, 0);
  \draw (0.5, 0) -- (1, 0) -- (1, 0.5) -- (0.5, 0.5);
  \draw (1, 0) -- (1.5, 0) -- (1.5, 0.5) -- (1, 0.5);
  \draw (1.5, 0) -- (2, 0) -- (2, 0.5) -- (1.5, 0.5);
  \draw (2, 0) -- (2.5, 0) -- (2.5, 0.5) -- (2, 0.5);
  \draw (2.5, 0) -- (3, 0) -- (3, 0.5) -- (2.5, 0.5);
  \draw (3, 0) -- (3.5, 0) -- (3.5, 0.5) -- (3, 0.5);
  \draw (3.5, 0) -- (4, 0) -- (4, 0.5) -- (3.5, 0.5);

\end{diagram}

Let's also say that each side register is labeled with a number. We'll call this the \vocab{address} of the side register. It's kind of like the address to a house. When we are looking for a particular house on a street, we look for its house number. Likewise here. This side register has an address, and we can pick out this particular side register by its number. Here's a picture, with the address attached to it:

\begin{diagram}

  \node at (-0.75, 0.25) {\textsf{0111}};
  \draw (0, 0) -- (-1.5, 0) -- (-1.5, 0.5) -- (0, 0.5);
  \draw[color=gray] (0.2, 0) -- (0.2, 0.5) -- (0.3, 0.5) -- (0.3, 0) -- (0.2, 0);
  \draw[color=gray] (0.7, 0) -- (0.7, 0.5) -- (0.8, 0.5) -- (0.8, 0) -- (0.7, 0);
  \draw[color=gray] (1.2, 0) -- (1.2, 0.5) -- (1.3, 0.5) -- (1.3, 0) -- (1.2, 0);
  \draw[color=gray] (1.7, 0) -- (1.7, 0.5) -- (1.8, 0.5) -- (1.8, 0) -- (1.7, 0);
  \draw[color=gray] (2.2, 0) -- (2.2, 0.5) -- (2.3, 0.5) -- (2.3, 0) -- (2.2, 0);
  \draw[color=gray] (2.7, 0) -- (2.7, 0.5) -- (2.8, 0.5) -- (2.8, 0) -- (2.7, 0);
  \draw[color=gray] (3.2, 0) -- (3.2, 0.5) -- (3.3, 0.5) -- (3.3, 0) -- (3.2, 0);
  \draw[color=gray] (3.7, 0) -- (3.7, 0.5) -- (3.8, 0.5) -- (3.8, 0) -- (3.7, 0);
  \draw (0, 0) -- (0.5, 0) -- (0.5, 0.5) -- (0, 0.5) -- (0, 0);
  \draw (0.5, 0) -- (1, 0) -- (1, 0.5) -- (0.5, 0.5);
  \draw (1, 0) -- (1.5, 0) -- (1.5, 0.5) -- (1, 0.5);
  \draw (1.5, 0) -- (2, 0) -- (2, 0.5) -- (1.5, 0.5);
  \draw (2, 0) -- (2.5, 0) -- (2.5, 0.5) -- (2, 0.5);
  \draw (2.5, 0) -- (3, 0) -- (3, 0.5) -- (2.5, 0.5);
  \draw (3, 0) -- (3.5, 0) -- (3.5, 0.5) -- (3, 0.5);
  \draw (3.5, 0) -- (4, 0) -- (4, 0.5) -- (3.5, 0.5);
  
  \draw (-1.4, -0.25) -- (-1.4, -0.5) -- (-0.1, -0.5) -- (-0.1, -0.25);
  \draw[->] (-0.75, -0.5) -- (-1.25, -1);
  \node at (-1.5, -1.4) {the address};
  \node at (-1.5, -2) {of the register};
  \node at (-1.5, -2.5) {in memory};
  
  \draw (0.1, -0.25) -- (0.1, -0.5) -- (3.9, -0.5) -- (3.9, -0.25);
  \draw[->] (2, -0.5) -- (2.25, -1);
  \node at (2.5, -1.5) {the register where};
  \node at (2.5, -1.95) {we can store};
  \node at (2.5, -2.5) {one byte of digits};

\end{diagram}

Next, let's imagine that we stack a number of these addressed side registers on top of each other. For our purposes here, let's suppose that we have just eight side registers, stacked on top of each other. Let's say that the register on the top of the stack has address ``$0$'' (in binary: $0000$), the next one underneath it has address ``$1$'' (in binary: $0001$), then the next has address ``$2$'' (in binary: $0010$), and so on. Like this:

\begin{diagram}

  \node at (-0.75, 3.75) {\textsf{0000}};
  \draw (0, 3.5) -- (-1.5, 3.5) -- (-1.5, 4) -- (0, 4);
  \draw[color=gray] (0.2, 3.5) -- (0.2, 4) -- (0.3, 4) -- (0.3, 3.5) -- (0.2, 3.5);
  \draw[color=gray] (0.7, 3.5) -- (0.7, 4) -- (0.8, 4) -- (0.8, 3.5) -- (0.7, 3.5);
  \draw[color=gray] (1.2, 3.5) -- (1.2, 4) -- (1.3, 4) -- (1.3, 3.5) -- (1.2, 3.5);
  \draw[color=gray] (1.7, 3.5) -- (1.7, 4) -- (1.8, 4) -- (1.8, 3.5) -- (1.7, 3.5);
  \draw[color=gray] (2.2, 3.5) -- (2.2, 4) -- (2.3, 4) -- (2.3, 3.5) -- (2.2, 3.5);
  \draw[color=gray] (2.7, 3.5) -- (2.7, 4) -- (2.8, 4) -- (2.8, 3.5) -- (2.7, 3.5);
  \draw[color=gray] (3.2, 3.5) -- (3.2, 4) -- (3.3, 4) -- (3.3, 3.5) -- (3.2, 3.5);
  \draw[color=gray] (3.7, 3.5) -- (3.7, 4) -- (3.8, 4) -- (3.8, 3.5) -- (3.7, 3.5);
  \draw (0, 3.5) -- (0.5, 3.5) -- (0.5, 4) -- (0, 4) -- (0, 3.5);
  \draw (0.5, 3.5) -- (1, 3.5) -- (1, 4) -- (0.5, 4);
  \draw (1, 3.5) -- (1.5, 3.5) -- (1.5, 4) -- (1, 4);
  \draw (1.5, 3.5) -- (2, 3.5) -- (2, 4) -- (1.5, 4);
  \draw (2, 3.5) -- (2.5, 3.5) -- (2.5, 4) -- (2, 4);
  \draw (2.5, 3.5) -- (3, 3.5) -- (3, 4) -- (2.5, 4);
  \draw (3, 3.5) -- (3.5, 3.5) -- (3.5, 4) -- (3, 4);
  \draw (3.5, 3.5) -- (4, 3.5) -- (4, 4) -- (3.5, 4);
  
  \node at (-0.75, 3.25) {\textsf{0001}};
  \draw (0, 3) -- (-1.5, 3) -- (-1.5, 3.5) -- (0, 3.5);
  \draw[color=gray] (0.2, 3) -- (0.2, 3.5) -- (0.3, 3.5) -- (0.3, 3) -- (0.2, 3);
  \draw[color=gray] (0.7, 3) -- (0.7, 3.5) -- (0.8, 3.5) -- (0.8, 3) -- (0.7, 3);
  \draw[color=gray] (1.2, 3) -- (1.2, 3.5) -- (1.3, 3.5) -- (1.3, 3) -- (1.2, 3);
  \draw[color=gray] (1.7, 3) -- (1.7, 3.5) -- (1.8, 3.5) -- (1.8, 3) -- (1.7, 3);
  \draw[color=gray] (2.2, 3) -- (2.2, 3.5) -- (2.3, 3.5) -- (2.3, 3) -- (2.2, 3);
  \draw[color=gray] (2.7, 3) -- (2.7, 3.5) -- (2.8, 3.5) -- (2.8, 3) -- (2.7, 3);
  \draw[color=gray] (3.2, 3) -- (3.2, 3.5) -- (3.3, 3.5) -- (3.3, 3) -- (3.2, 3);
  \draw[color=gray] (3.7, 3) -- (3.7, 3.5) -- (3.8, 3.5) -- (3.8, 3) -- (3.7, 3);
  \draw (0, 3) -- (0.5, 3) -- (0.5, 3.5) -- (0, 3.5) -- (0, 3);
  \draw (0.5, 3) -- (1, 3) -- (1, 3.5) -- (0.5, 3.5);
  \draw (1, 3) -- (1.5, 3) -- (1.5, 3.5) -- (1, 3.5);
  \draw (1.5, 3) -- (2, 3) -- (2, 3.5) -- (1.5, 3.5);
  \draw (2, 3) -- (2.5, 3) -- (2.5, 3.5) -- (2, 3.5);
  \draw (2.5, 3) -- (3, 3) -- (3, 3.5) -- (2.5, 3.5);
  \draw (3, 3) -- (3.5, 3) -- (3.5, 3.5) -- (3, 3.5);
  \draw (3.5, 3) -- (4, 3) -- (4, 3.5) -- (3.5, 3.5);

  \node at (-0.75, 2.75) {\textsf{0010}};
  \draw (0, 2.5) -- (-1.5, 2.5) -- (-1.5, 3) -- (0, 3);
  \draw[color=gray] (0.2, 2.5) -- (0.2, 3) -- (0.3, 3) -- (0.3, 2.5) -- (0.2, 2.5);
  \draw[color=gray] (0.7, 2.5) -- (0.7, 3) -- (0.8, 3) -- (0.8, 2.5) -- (0.7, 2.5);
  \draw[color=gray] (1.2, 2.5) -- (1.2, 3) -- (1.3, 3) -- (1.3, 2.5) -- (1.2, 2.5);
  \draw[color=gray] (1.7, 2.5) -- (1.7, 3) -- (1.8, 3) -- (1.8, 2.5) -- (1.7, 2.5);
  \draw[color=gray] (2.2, 2.5) -- (2.2, 3) -- (2.3, 3) -- (2.3, 2.5) -- (2.2, 2.5);
  \draw[color=gray] (2.7, 2.5) -- (2.7, 3) -- (2.8, 3) -- (2.8, 2.5) -- (2.7, 2.5);
  \draw[color=gray] (3.2, 2.5) -- (3.2, 3) -- (3.3, 3) -- (3.3, 2.5) -- (3.2, 2.5);
  \draw[color=gray] (3.7, 2.5) -- (3.7, 3) -- (3.8, 3) -- (3.8, 2.5) -- (3.7, 2.5);
  \draw (0, 2.5) -- (0.5, 2.5) -- (0.5, 3) -- (0, 3) -- (0, 2.5);
  \draw (0.5, 2.5) -- (1, 2.5) -- (1, 3) -- (0.5, 3);
  \draw (1, 2.5) -- (1.5, 2.5) -- (1.5, 3) -- (1, 3);
  \draw (1.5, 2.5) -- (2, 2.5) -- (2, 3) -- (1.5, 3);
  \draw (2, 2.5) -- (2.5, 2.5) -- (2.5, 3) -- (2, 3);
  \draw (2.5, 2.5) -- (3, 2.5) -- (3, 3) -- (2.5, 3);
  \draw (3, 2.5) -- (3.5, 2.5) -- (3.5, 3) -- (3, 3);
  \draw (3.5, 2.5) -- (4, 2.5) -- (4, 3) -- (3.5, 3);
  
  \node at (-0.75, 2.25) {\textsf{0011}};
  \draw (0, 2) -- (-1.5, 2) -- (-1.5, 2.5) -- (0, 2.5);
  \draw[color=gray] (0.2, 2) -- (0.2, 2.5) -- (0.3, 2.5) -- (0.3, 2) -- (0.2, 2);
  \draw[color=gray] (0.7, 2) -- (0.7, 2.5) -- (0.8, 2.5) -- (0.8, 2) -- (0.7, 2);
  \draw[color=gray] (1.2, 2) -- (1.2, 2.5) -- (1.3, 2.5) -- (1.3, 2) -- (1.2, 2);
  \draw[color=gray] (1.7, 2) -- (1.7, 2.5) -- (1.8, 2.5) -- (1.8, 2) -- (1.7, 2);
  \draw[color=gray] (2.2, 2) -- (2.2, 2.5) -- (2.3, 2.5) -- (2.3, 2) -- (2.2, 2);
  \draw[color=gray] (2.7, 2) -- (2.7, 2.5) -- (2.8, 2.5) -- (2.8, 2) -- (2.7, 2);
  \draw[color=gray] (3.2, 2) -- (3.2, 2.5) -- (3.3, 2.5) -- (3.3, 2) -- (3.2, 2);
  \draw[color=gray] (3.7, 2) -- (3.7, 2.5) -- (3.8, 2.5) -- (3.8, 2) -- (3.7, 2);
  \draw (0, 2) -- (0.5, 2) -- (0.5, 2.5) -- (0, 2.5) -- (0, 2);
  \draw (0.5, 2) -- (1, 2) -- (1, 2.5) -- (0.5, 2.5);
  \draw (1, 2) -- (1.5, 2) -- (1.5, 2.5) -- (1, 2.5);
  \draw (1.5, 2) -- (2, 2) -- (2, 2.5) -- (1.5, 2.5);
  \draw (2, 2) -- (2.5, 2) -- (2.5, 2.5) -- (2, 2.5);
  \draw (2.5, 2) -- (3, 2) -- (3, 2.5) -- (2.5, 2.5);
  \draw (3, 2) -- (3.5, 2) -- (3.5, 2.5) -- (3, 2.5);
  \draw (3.5, 2) -- (4, 2) -- (4, 2.5) -- (3.5, 2.5);

  \node at (-0.75, 1.75) {\textsf{0100}};
  \draw (0, 1.5) -- (-1.5, 1.5) -- (-1.5, 2) -- (0, 2);
  \draw[color=gray] (0.2, 1.5) -- (0.2, 2) -- (0.3, 2) -- (0.3, 1.5) -- (0.2, 1.5);
  \draw[color=gray] (0.7, 1.5) -- (0.7, 2) -- (0.8, 2) -- (0.8, 1.5) -- (0.7, 1.5);
  \draw[color=gray] (1.2, 1.5) -- (1.2, 2) -- (1.3, 2) -- (1.3, 1.5) -- (1.2, 1.5);
  \draw[color=gray] (1.7, 1.5) -- (1.7, 2) -- (1.8, 2) -- (1.8, 1.5) -- (1.7, 1.5);
  \draw[color=gray] (2.2, 1.5) -- (2.2, 2) -- (2.3, 2) -- (2.3, 1.5) -- (2.2, 1.5);
  \draw[color=gray] (2.7, 1.5) -- (2.7, 2) -- (2.8, 2) -- (2.8, 1.5) -- (2.7, 1.5);
  \draw[color=gray] (3.2, 1.5) -- (3.2, 2) -- (3.3, 2) -- (3.3, 1.5) -- (3.2, 1.5);
  \draw[color=gray] (3.7, 1.5) -- (3.7, 2) -- (3.8, 2) -- (3.8, 1.5) -- (3.7, 1.5);
  \draw (0, 1.5) -- (0.5, 1.5) -- (0.5, 2) -- (0, 2) -- (0, 1.5);
  \draw (0.5, 1.5) -- (1, 1.5) -- (1, 2) -- (0.5, 2);
  \draw (1, 1.5) -- (1.5, 1.5) -- (1.5, 2) -- (1, 2);
  \draw (1.5, 1.5) -- (2, 1.5) -- (2, 2) -- (1.5, 2);
  \draw (2, 1.5) -- (2.5, 1.5) -- (2.5, 2) -- (2, 2);
  \draw (2.5, 1.5) -- (3, 1.5) -- (3, 2) -- (2.5, 2);
  \draw (3, 1.5) -- (3.5, 1.5) -- (3.5, 2) -- (3, 2);
  \draw (3.5, 1.5) -- (4, 1.5) -- (4, 2) -- (3.5, 2);
  
  \node at (-0.75, 1.25) {\textsf{0101}};
  \draw (0, 1) -- (-1.5, 1) -- (-1.5, 1.5) -- (0, 1.5);
  \draw[color=gray] (0.2, 1) -- (0.2, 1.5) -- (0.3, 1.5) -- (0.3, 1) -- (0.2, 1);
  \draw[color=gray] (0.7, 1) -- (0.7, 1.5) -- (0.8, 1.5) -- (0.8, 1) -- (0.7, 1);
  \draw[color=gray] (1.2, 1) -- (1.2, 1.5) -- (1.3, 1.5) -- (1.3, 1) -- (1.2, 1);
  \draw[color=gray] (1.7, 1) -- (1.7, 1.5) -- (1.8, 1.5) -- (1.8, 1) -- (1.7, 1);
  \draw[color=gray] (2.2, 1) -- (2.2, 1.5) -- (2.3, 1.5) -- (2.3, 1) -- (2.2, 1);
  \draw[color=gray] (2.7, 1) -- (2.7, 1.5) -- (2.8, 1.5) -- (2.8, 1) -- (2.7, 1);
  \draw[color=gray] (3.2, 1) -- (3.2, 1.5) -- (3.3, 1.5) -- (3.3, 1) -- (3.2, 1);
  \draw[color=gray] (3.7, 1) -- (3.7, 1.5) -- (3.8, 1.5) -- (3.8, 1) -- (3.7, 1);
  \draw (0, 1) -- (0.5, 1) -- (0.5, 1.5) -- (0, 1.5) -- (0, 1);
  \draw (0.5, 1) -- (1, 1) -- (1, 1.5) -- (0.5, 1.5);
  \draw (1, 1) -- (1.5, 1) -- (1.5, 1.5) -- (1, 1.5);
  \draw (1.5, 1) -- (2, 1) -- (2, 1.5) -- (1.5, 1.5);
  \draw (2, 1) -- (2.5, 1) -- (2.5, 1.5) -- (2, 1.5);
  \draw (2.5, 1) -- (3, 1) -- (3, 1.5) -- (2.5, 1.5);
  \draw (3, 1) -- (3.5, 1) -- (3.5, 1.5) -- (3, 1.5);
  \draw (3.5, 1) -- (4, 1) -- (4, 1.5) -- (3.5, 1.5);

  \node at (-0.75, 0.75) {\textsf{0110}};
  \draw (0, 0.5) -- (-1.5, 0.5) -- (-1.5, 1) -- (0, 1);
  \draw[color=gray] (0.2, 0.5) -- (0.2, 1) -- (0.3, 1) -- (0.3, 0.5) -- (0.2, 0.5);
  \draw[color=gray] (0.7, 0.5) -- (0.7, 1) -- (0.8, 1) -- (0.8, 0.5) -- (0.7, 0.5);
  \draw[color=gray] (1.2, 0.5) -- (1.2, 1) -- (1.3, 1) -- (1.3, 0.5) -- (1.2, 0.5);
  \draw[color=gray] (1.7, 0.5) -- (1.7, 1) -- (1.8, 1) -- (1.8, 0.5) -- (1.7, 0.5);
  \draw[color=gray] (2.2, 0.5) -- (2.2, 1) -- (2.3, 1) -- (2.3, 0.5) -- (2.2, 0.5);
  \draw[color=gray] (2.7, 0.5) -- (2.7, 1) -- (2.8, 1) -- (2.8, 0.5) -- (2.7, 0.5);
  \draw[color=gray] (3.2, 0.5) -- (3.2, 1) -- (3.3, 1) -- (3.3, 0.5) -- (3.2, 0.5);
  \draw[color=gray] (3.7, 0.5) -- (3.7, 1) -- (3.8, 1) -- (3.8, 0.5) -- (3.7, 0.5);
  \draw (0, 0.5) -- (0.5, 0.5) -- (0.5, 1) -- (0, 1) -- (0, 0.5);
  \draw (0.5, 0.5) -- (1, 0.5) -- (1, 1) -- (0.5, 1);
  \draw (1, 0.5) -- (1.5, 0.5) -- (1.5, 1) -- (1, 1);
  \draw (1.5, 0.5) -- (2, 0.5) -- (2, 1) -- (1.5, 1);
  \draw (2, 0.5) -- (2.5, 0.5) -- (2.5, 1) -- (2, 1);
  \draw (2.5, 0.5) -- (3, 0.5) -- (3, 1) -- (2.5, 1);
  \draw (3, 0.5) -- (3.5, 0.5) -- (3.5, 1) -- (3, 1);
  \draw (3.5, 0.5) -- (4, 0.5) -- (4, 1) -- (3.5, 1);
  
  \node at (-0.75, 0.25) {\textsf{0111}};
  \draw (0, 0) -- (-1.5, 0) -- (-1.5, 0.5) -- (0, 0.5);
  \draw[color=gray] (0.2, 0) -- (0.2, 0.5) -- (0.3, 0.5) -- (0.3, 0) -- (0.2, 0);
  \draw[color=gray] (0.7, 0) -- (0.7, 0.5) -- (0.8, 0.5) -- (0.8, 0) -- (0.7, 0);
  \draw[color=gray] (1.2, 0) -- (1.2, 0.5) -- (1.3, 0.5) -- (1.3, 0) -- (1.2, 0);
  \draw[color=gray] (1.7, 0) -- (1.7, 0.5) -- (1.8, 0.5) -- (1.8, 0) -- (1.7, 0);
  \draw[color=gray] (2.2, 0) -- (2.2, 0.5) -- (2.3, 0.5) -- (2.3, 0) -- (2.2, 0);
  \draw[color=gray] (2.7, 0) -- (2.7, 0.5) -- (2.8, 0.5) -- (2.8, 0) -- (2.7, 0);
  \draw[color=gray] (3.2, 0) -- (3.2, 0.5) -- (3.3, 0.5) -- (3.3, 0) -- (3.2, 0);
  \draw[color=gray] (3.7, 0) -- (3.7, 0.5) -- (3.8, 0.5) -- (3.8, 0) -- (3.7, 0);
  \draw (0, 0) -- (0.5, 0) -- (0.5, 0.5) -- (0, 0.5) -- (0, 0);
  \draw (0.5, 0) -- (1, 0) -- (1, 0.5) -- (0.5, 0.5);
  \draw (1, 0) -- (1.5, 0) -- (1.5, 0.5) -- (1, 0.5);
  \draw (1.5, 0) -- (2, 0) -- (2, 0.5) -- (1.5, 0.5);
  \draw (2, 0) -- (2.5, 0) -- (2.5, 0.5) -- (2, 0.5);
  \draw (2.5, 0) -- (3, 0) -- (3, 0.5) -- (2.5, 0.5);
  \draw (3, 0) -- (3.5, 0) -- (3.5, 0.5) -- (3, 0.5);
  \draw (3.5, 0) -- (4, 0) -- (4, 0.5) -- (3.5, 0.5);

  \draw (-1.4, -0.25) -- (-1.4, -0.5) -- (-0.1, -0.5) -- (-0.1, -0.25);
  \draw[->] (-0.75, -0.5) -- (-1.25, -1);
  \node at (-1.5, -1.4) {addresses};
  
  \draw (0.1, -0.25) -- (0.1, -0.5) -- (3.9, -0.5) -- (3.9, -0.25);
  \draw[->] (2, -0.5) -- (2.25, -1);
  \node at (2.5, -1.5) {registers};

\end{diagram}

In essence, this gives us a memory bank. It is a stack of ``slots'' (side registers) that we can put numbers in, where each ``slot'' has an address. For instance, we can put the number five (in binary: $00001010$) in the slot addressed as ``$0000$,'' and we can put the number seventy-seven (in binary: $01001101$) in the slot addressed as ``$0011$'':

\begin{diagram}

  \node at (6, 3.75) {$00001010$};
  \draw[<-] (4.25, 3.75) -- (5, 3.75);
  
  \node at (-0.75, 3.75) {\textsf{0000}};
  \draw (0, 3.5) -- (-1.5, 3.5) -- (-1.5, 4) -- (0, 4);
  \draw[color=gray] (0.2, 3.5) -- (0.2, 4) -- (0.3, 4) -- (0.3, 3.5) -- (0.2, 3.5);
  \draw[color=gray] (0.7, 3.5) -- (0.7, 4) -- (0.8, 4) -- (0.8, 3.5) -- (0.7, 3.5);
  \draw[color=gray] (1.2, 3.5) -- (1.2, 4) -- (1.3, 4) -- (1.3, 3.5) -- (1.2, 3.5);
  \draw[color=gray] (1.7, 3.5) -- (1.7, 4) -- (1.8, 4) -- (1.8, 3.5) -- (1.7, 3.5);
  \draw[color=gray,fill=black] (2.2, 3.5) -- (2.2, 4) -- (2.3, 4) -- (2.3, 3.5) -- (2.2, 3.5);
  \draw[color=gray] (2.7, 3.5) -- (2.7, 4) -- (2.8, 4) -- (2.8, 3.5) -- (2.7, 3.5);
  \draw[color=gray,fill=black] (3.2, 3.5) -- (3.2, 4) -- (3.3, 4) -- (3.3, 3.5) -- (3.2, 3.5);
  \draw[color=gray] (3.7, 3.5) -- (3.7, 4) -- (3.8, 4) -- (3.8, 3.5) -- (3.7, 3.5);
  \draw (0, 3.5) -- (0.5, 3.5) -- (0.5, 4) -- (0, 4) -- (0, 3.5);
  \draw (0.5, 3.5) -- (1, 3.5) -- (1, 4) -- (0.5, 4);
  \draw (1, 3.5) -- (1.5, 3.5) -- (1.5, 4) -- (1, 4);
  \draw (1.5, 3.5) -- (2, 3.5) -- (2, 4) -- (1.5, 4);
  \draw (2, 3.5) -- (2.5, 3.5) -- (2.5, 4) -- (2, 4);
  \draw (2.5, 3.5) -- (3, 3.5) -- (3, 4) -- (2.5, 4);
  \draw (3, 3.5) -- (3.5, 3.5) -- (3.5, 4) -- (3, 4);
  \draw (3.5, 3.5) -- (4, 3.5) -- (4, 4) -- (3.5, 4);
  
  \node at (-0.75, 3.25) {\textsf{0001}};
  \draw (0, 3) -- (-1.5, 3) -- (-1.5, 3.5) -- (0, 3.5);
  \draw[color=gray] (0.2, 3) -- (0.2, 3.5) -- (0.3, 3.5) -- (0.3, 3) -- (0.2, 3);
  \draw[color=gray] (0.7, 3) -- (0.7, 3.5) -- (0.8, 3.5) -- (0.8, 3) -- (0.7, 3);
  \draw[color=gray] (1.2, 3) -- (1.2, 3.5) -- (1.3, 3.5) -- (1.3, 3) -- (1.2, 3);
  \draw[color=gray] (1.7, 3) -- (1.7, 3.5) -- (1.8, 3.5) -- (1.8, 3) -- (1.7, 3);
  \draw[color=gray] (2.2, 3) -- (2.2, 3.5) -- (2.3, 3.5) -- (2.3, 3) -- (2.2, 3);
  \draw[color=gray] (2.7, 3) -- (2.7, 3.5) -- (2.8, 3.5) -- (2.8, 3) -- (2.7, 3);
  \draw[color=gray] (3.2, 3) -- (3.2, 3.5) -- (3.3, 3.5) -- (3.3, 3) -- (3.2, 3);
  \draw[color=gray] (3.7, 3) -- (3.7, 3.5) -- (3.8, 3.5) -- (3.8, 3) -- (3.7, 3);
  \draw (0, 3) -- (0.5, 3) -- (0.5, 3.5) -- (0, 3.5) -- (0, 3);
  \draw (0.5, 3) -- (1, 3) -- (1, 3.5) -- (0.5, 3.5);
  \draw (1, 3) -- (1.5, 3) -- (1.5, 3.5) -- (1, 3.5);
  \draw (1.5, 3) -- (2, 3) -- (2, 3.5) -- (1.5, 3.5);
  \draw (2, 3) -- (2.5, 3) -- (2.5, 3.5) -- (2, 3.5);
  \draw (2.5, 3) -- (3, 3) -- (3, 3.5) -- (2.5, 3.5);
  \draw (3, 3) -- (3.5, 3) -- (3.5, 3.5) -- (3, 3.5);
  \draw (3.5, 3) -- (4, 3) -- (4, 3.5) -- (3.5, 3.5);

  \node at (-0.75, 2.75) {\textsf{0010}};
  \draw (0, 2.5) -- (-1.5, 2.5) -- (-1.5, 3) -- (0, 3);
  \draw[color=gray] (0.2, 2.5) -- (0.2, 3) -- (0.3, 3) -- (0.3, 2.5) -- (0.2, 2.5);
  \draw[color=gray] (0.7, 2.5) -- (0.7, 3) -- (0.8, 3) -- (0.8, 2.5) -- (0.7, 2.5);
  \draw[color=gray] (1.2, 2.5) -- (1.2, 3) -- (1.3, 3) -- (1.3, 2.5) -- (1.2, 2.5);
  \draw[color=gray] (1.7, 2.5) -- (1.7, 3) -- (1.8, 3) -- (1.8, 2.5) -- (1.7, 2.5);
  \draw[color=gray] (2.2, 2.5) -- (2.2, 3) -- (2.3, 3) -- (2.3, 2.5) -- (2.2, 2.5);
  \draw[color=gray] (2.7, 2.5) -- (2.7, 3) -- (2.8, 3) -- (2.8, 2.5) -- (2.7, 2.5);
  \draw[color=gray] (3.2, 2.5) -- (3.2, 3) -- (3.3, 3) -- (3.3, 2.5) -- (3.2, 2.5);
  \draw[color=gray] (3.7, 2.5) -- (3.7, 3) -- (3.8, 3) -- (3.8, 2.5) -- (3.7, 2.5);
  \draw (0, 2.5) -- (0.5, 2.5) -- (0.5, 3) -- (0, 3) -- (0, 2.5);
  \draw (0.5, 2.5) -- (1, 2.5) -- (1, 3) -- (0.5, 3);
  \draw (1, 2.5) -- (1.5, 2.5) -- (1.5, 3) -- (1, 3);
  \draw (1.5, 2.5) -- (2, 2.5) -- (2, 3) -- (1.5, 3);
  \draw (2, 2.5) -- (2.5, 2.5) -- (2.5, 3) -- (2, 3);
  \draw (2.5, 2.5) -- (3, 2.5) -- (3, 3) -- (2.5, 3);
  \draw (3, 2.5) -- (3.5, 2.5) -- (3.5, 3) -- (3, 3);
  \draw (3.5, 2.5) -- (4, 2.5) -- (4, 3) -- (3.5, 3);

  \node at (6, 2.25) {$01001101$};
  \draw[<-] (4.25, 2.25) -- (5, 2.25);
  
  \node at (-0.75, 2.25) {\textsf{0011}};
  \draw (0, 2) -- (-1.5, 2) -- (-1.5, 2.5) -- (0, 2.5);
  \draw[color=gray] (0.2, 2) -- (0.2, 2.5) -- (0.3, 2.5) -- (0.3, 2) -- (0.2, 2);
  \draw[color=gray,fill=black] (0.7, 2) -- (0.7, 2.5) -- (0.8, 2.5) -- (0.8, 2) -- (0.7, 2);
  \draw[color=gray] (1.2, 2) -- (1.2, 2.5) -- (1.3, 2.5) -- (1.3, 2) -- (1.2, 2);
  \draw[color=gray] (1.7, 2) -- (1.7, 2.5) -- (1.8, 2.5) -- (1.8, 2) -- (1.7, 2);
  \draw[color=gray,fill=black] (2.2, 2) -- (2.2, 2.5) -- (2.3, 2.5) -- (2.3, 2) -- (2.2, 2);
  \draw[color=gray,fill=black] (2.7, 2) -- (2.7, 2.5) -- (2.8, 2.5) -- (2.8, 2) -- (2.7, 2);
  \draw[color=gray] (3.2, 2) -- (3.2, 2.5) -- (3.3, 2.5) -- (3.3, 2) -- (3.2, 2);
  \draw[color=gray,fill=black] (3.7, 2) -- (3.7, 2.5) -- (3.8, 2.5) -- (3.8, 2) -- (3.7, 2);
  \draw (0, 2) -- (0.5, 2) -- (0.5, 2.5) -- (0, 2.5) -- (0, 2);
  \draw (0.5, 2) -- (1, 2) -- (1, 2.5) -- (0.5, 2.5);
  \draw (1, 2) -- (1.5, 2) -- (1.5, 2.5) -- (1, 2.5);
  \draw (1.5, 2) -- (2, 2) -- (2, 2.5) -- (1.5, 2.5);
  \draw (2, 2) -- (2.5, 2) -- (2.5, 2.5) -- (2, 2.5);
  \draw (2.5, 2) -- (3, 2) -- (3, 2.5) -- (2.5, 2.5);
  \draw (3, 2) -- (3.5, 2) -- (3.5, 2.5) -- (3, 2.5);
  \draw (3.5, 2) -- (4, 2) -- (4, 2.5) -- (3.5, 2.5);

  \node at (-0.75, 1.75) {\textsf{0100}};
  \draw (0, 1.5) -- (-1.5, 1.5) -- (-1.5, 2) -- (0, 2);
  \draw[color=gray] (0.2, 1.5) -- (0.2, 2) -- (0.3, 2) -- (0.3, 1.5) -- (0.2, 1.5);
  \draw[color=gray] (0.7, 1.5) -- (0.7, 2) -- (0.8, 2) -- (0.8, 1.5) -- (0.7, 1.5);
  \draw[color=gray] (1.2, 1.5) -- (1.2, 2) -- (1.3, 2) -- (1.3, 1.5) -- (1.2, 1.5);
  \draw[color=gray] (1.7, 1.5) -- (1.7, 2) -- (1.8, 2) -- (1.8, 1.5) -- (1.7, 1.5);
  \draw[color=gray] (2.2, 1.5) -- (2.2, 2) -- (2.3, 2) -- (2.3, 1.5) -- (2.2, 1.5);
  \draw[color=gray] (2.7, 1.5) -- (2.7, 2) -- (2.8, 2) -- (2.8, 1.5) -- (2.7, 1.5);
  \draw[color=gray] (3.2, 1.5) -- (3.2, 2) -- (3.3, 2) -- (3.3, 1.5) -- (3.2, 1.5);
  \draw[color=gray] (3.7, 1.5) -- (3.7, 2) -- (3.8, 2) -- (3.8, 1.5) -- (3.7, 1.5);
  \draw (0, 1.5) -- (0.5, 1.5) -- (0.5, 2) -- (0, 2) -- (0, 1.5);
  \draw (0.5, 1.5) -- (1, 1.5) -- (1, 2) -- (0.5, 2);
  \draw (1, 1.5) -- (1.5, 1.5) -- (1.5, 2) -- (1, 2);
  \draw (1.5, 1.5) -- (2, 1.5) -- (2, 2) -- (1.5, 2);
  \draw (2, 1.5) -- (2.5, 1.5) -- (2.5, 2) -- (2, 2);
  \draw (2.5, 1.5) -- (3, 1.5) -- (3, 2) -- (2.5, 2);
  \draw (3, 1.5) -- (3.5, 1.5) -- (3.5, 2) -- (3, 2);
  \draw (3.5, 1.5) -- (4, 1.5) -- (4, 2) -- (3.5, 2);
  
  \node at (-0.75, 1.25) {\textsf{0101}};
  \draw (0, 1) -- (-1.5, 1) -- (-1.5, 1.5) -- (0, 1.5);
  \draw[color=gray] (0.2, 1) -- (0.2, 1.5) -- (0.3, 1.5) -- (0.3, 1) -- (0.2, 1);
  \draw[color=gray] (0.7, 1) -- (0.7, 1.5) -- (0.8, 1.5) -- (0.8, 1) -- (0.7, 1);
  \draw[color=gray] (1.2, 1) -- (1.2, 1.5) -- (1.3, 1.5) -- (1.3, 1) -- (1.2, 1);
  \draw[color=gray] (1.7, 1) -- (1.7, 1.5) -- (1.8, 1.5) -- (1.8, 1) -- (1.7, 1);
  \draw[color=gray] (2.2, 1) -- (2.2, 1.5) -- (2.3, 1.5) -- (2.3, 1) -- (2.2, 1);
  \draw[color=gray] (2.7, 1) -- (2.7, 1.5) -- (2.8, 1.5) -- (2.8, 1) -- (2.7, 1);
  \draw[color=gray] (3.2, 1) -- (3.2, 1.5) -- (3.3, 1.5) -- (3.3, 1) -- (3.2, 1);
  \draw[color=gray] (3.7, 1) -- (3.7, 1.5) -- (3.8, 1.5) -- (3.8, 1) -- (3.7, 1);
  \draw (0, 1) -- (0.5, 1) -- (0.5, 1.5) -- (0, 1.5) -- (0, 1);
  \draw (0.5, 1) -- (1, 1) -- (1, 1.5) -- (0.5, 1.5);
  \draw (1, 1) -- (1.5, 1) -- (1.5, 1.5) -- (1, 1.5);
  \draw (1.5, 1) -- (2, 1) -- (2, 1.5) -- (1.5, 1.5);
  \draw (2, 1) -- (2.5, 1) -- (2.5, 1.5) -- (2, 1.5);
  \draw (2.5, 1) -- (3, 1) -- (3, 1.5) -- (2.5, 1.5);
  \draw (3, 1) -- (3.5, 1) -- (3.5, 1.5) -- (3, 1.5);
  \draw (3.5, 1) -- (4, 1) -- (4, 1.5) -- (3.5, 1.5);

  \node at (-0.75, 0.75) {\textsf{0110}};
  \draw (0, 0.5) -- (-1.5, 0.5) -- (-1.5, 1) -- (0, 1);
  \draw[color=gray] (0.2, 0.5) -- (0.2, 1) -- (0.3, 1) -- (0.3, 0.5) -- (0.2, 0.5);
  \draw[color=gray] (0.7, 0.5) -- (0.7, 1) -- (0.8, 1) -- (0.8, 0.5) -- (0.7, 0.5);
  \draw[color=gray] (1.2, 0.5) -- (1.2, 1) -- (1.3, 1) -- (1.3, 0.5) -- (1.2, 0.5);
  \draw[color=gray] (1.7, 0.5) -- (1.7, 1) -- (1.8, 1) -- (1.8, 0.5) -- (1.7, 0.5);
  \draw[color=gray] (2.2, 0.5) -- (2.2, 1) -- (2.3, 1) -- (2.3, 0.5) -- (2.2, 0.5);
  \draw[color=gray] (2.7, 0.5) -- (2.7, 1) -- (2.8, 1) -- (2.8, 0.5) -- (2.7, 0.5);
  \draw[color=gray] (3.2, 0.5) -- (3.2, 1) -- (3.3, 1) -- (3.3, 0.5) -- (3.2, 0.5);
  \draw[color=gray] (3.7, 0.5) -- (3.7, 1) -- (3.8, 1) -- (3.8, 0.5) -- (3.7, 0.5);
  \draw (0, 0.5) -- (0.5, 0.5) -- (0.5, 1) -- (0, 1) -- (0, 0.5);
  \draw (0.5, 0.5) -- (1, 0.5) -- (1, 1) -- (0.5, 1);
  \draw (1, 0.5) -- (1.5, 0.5) -- (1.5, 1) -- (1, 1);
  \draw (1.5, 0.5) -- (2, 0.5) -- (2, 1) -- (1.5, 1);
  \draw (2, 0.5) -- (2.5, 0.5) -- (2.5, 1) -- (2, 1);
  \draw (2.5, 0.5) -- (3, 0.5) -- (3, 1) -- (2.5, 1);
  \draw (3, 0.5) -- (3.5, 0.5) -- (3.5, 1) -- (3, 1);
  \draw (3.5, 0.5) -- (4, 0.5) -- (4, 1) -- (3.5, 1);
  
  \node at (-0.75, 0.25) {\textsf{0111}};
  \draw (0, 0) -- (-1.5, 0) -- (-1.5, 0.5) -- (0, 0.5);
  \draw[color=gray] (0.2, 0) -- (0.2, 0.5) -- (0.3, 0.5) -- (0.3, 0) -- (0.2, 0);
  \draw[color=gray] (0.7, 0) -- (0.7, 0.5) -- (0.8, 0.5) -- (0.8, 0) -- (0.7, 0);
  \draw[color=gray] (1.2, 0) -- (1.2, 0.5) -- (1.3, 0.5) -- (1.3, 0) -- (1.2, 0);
  \draw[color=gray] (1.7, 0) -- (1.7, 0.5) -- (1.8, 0.5) -- (1.8, 0) -- (1.7, 0);
  \draw[color=gray] (2.2, 0) -- (2.2, 0.5) -- (2.3, 0.5) -- (2.3, 0) -- (2.2, 0);
  \draw[color=gray] (2.7, 0) -- (2.7, 0.5) -- (2.8, 0.5) -- (2.8, 0) -- (2.7, 0);
  \draw[color=gray] (3.2, 0) -- (3.2, 0.5) -- (3.3, 0.5) -- (3.3, 0) -- (3.2, 0);
  \draw[color=gray] (3.7, 0) -- (3.7, 0.5) -- (3.8, 0.5) -- (3.8, 0) -- (3.7, 0);
  \draw (0, 0) -- (0.5, 0) -- (0.5, 0.5) -- (0, 0.5) -- (0, 0);
  \draw (0.5, 0) -- (1, 0) -- (1, 0.5) -- (0.5, 0.5);
  \draw (1, 0) -- (1.5, 0) -- (1.5, 0.5) -- (1, 0.5);
  \draw (1.5, 0) -- (2, 0) -- (2, 0.5) -- (1.5, 0.5);
  \draw (2, 0) -- (2.5, 0) -- (2.5, 0.5) -- (2, 0.5);
  \draw (2.5, 0) -- (3, 0) -- (3, 0.5) -- (2.5, 0.5);
  \draw (3, 0) -- (3.5, 0) -- (3.5, 0.5) -- (3, 0.5);
  \draw (3.5, 0) -- (4, 0) -- (4, 0.5) -- (3.5, 0.5);

\end{diagram}

\begin{aside}
  \begin{remark}
    Most modern computers have many more slots than we have here. Our imaginary machine only has eight slots, but that is only because we want to keep things simple, for pedagogical reasons. 
  \end{remark}
\end{aside}

Since this stack has eight slots, and each slot holds one byte (8-bits), this memory bank can hold a total of 8 bytes (or 64-bits). In computer speak, we can say that our imaginary machine here has 8-bytes of \vocab{memory} (or synonymously, we can say it has 8-bytes of \vocab{RAM}).


%%%%%%%%%%%%%%%%%%%%%%%%%%%%%%%%%%%%%%%%%
%%%%%%%%%%%%%%%%%%%%%%%%%%%%%%%%%%%%%%%%%
\section{The ALU}

\newthought{What about doing arithmetic and basic computation?} For this, every computer has a special little chip inside it called the \vocab{ALU} (short for the ``Arithmetic and Logic Unit''). There is no special magic to the ALU. It is just a bunch of little circuits that let us do some \emph{extremely basic} arithmetic with binary numbers.

Let's draw the ALU as a little box, like this:

\begin{diagram}

  \node at (-6, 1) {\textsf{ALU}};
  \draw (-8, 0) -- (-4, 0) -- (-4, 2) -- (-8, 2) -- (-8, 0);

\end{diagram}

The ALU also has attached to it three 1-bit registers. They are called ``\textsf{O},'' ``\textsf{N},'' and ``\textsf{Z}.'' Let's draw them like this:

\begin{diagram}

  \node at (-6, 1) {\textsf{ALU}};
  \draw (-8, 0) -- (-4, 0) -- (-4, 2) -- (-8, 2) -- (-8, 0);
  \node at (-3.375, 1.75) {\textsf{O}};
  \draw (-4, 1.75) -- (-3.75, 1.75);
  \draw (-3.75, 2) -- (-3, 2) -- (-3, 1.5) -- (-3.75, 1.5) -- (-3.75, 2);
  \draw[color=gray]
    (-2.8, 2) -- (-2.8, 1.5) -- (-2.7, 1.5) -- (-2.7, 2) -- (-2.8, 2);
  \draw (-3, 2) -- (-2.5, 2) -- (-2.5, 1.5) -- (-3, 1.5);
  \node at (-3.375, 1) {\textsf{N}};
  \draw (-4, 1) -- (-3.75, 1);
  \draw (-3.75, 1.25) -- (-3, 1.25) -- (-3, 0.75) -- (-3.75, 0.75) -- (-3.75, 1.25);
  \draw[color=gray]
    (-2.8, 1.25) -- (-2.8, 0.75) -- (-2.7, 0.75) -- (-2.7, 1.25) -- (-2.8, 1.25);
  \draw (-3, 1.25) -- (-2.5, 1.25) -- (-2.5, 0.75) -- (-3, 0.75);
  \node at (-3.375, 0.25) {\textsf{Z}};
  \draw (-4, 0.25) -- (-3.75, 0.25);
  \draw (-3.75, 0.5) -- (-3, 0.5) -- (-3, 0) -- (-3.75, 0) -- (-3.75, 0.5);
  \draw[color=gray]
    (-2.8, 0.5) -- (-2.8, 0) -- (-2.7, 0) -- (-2.7, 0.5) -- (-2.8, 0.5);
  \draw (-3, 0.5) -- (-2.5, 0.5) -- (-2.5, 0) -- (-3, 0);

\end{diagram}

We will talk about what these little registers are for in a moment.

In the most basic terms, the ALU works like this: we put two binary numbers into it, it adds them together, and it spits out the sum as a new binary number. For instance, suppose we want it to add ``one'' (in binary: $0001$) to ``two'' (in binary: $0010$), and so produce ``three'' (in binary: $0011$). We would feed the ALU ``$0001$'' and ``$0010$'' as inputs, it would add them together, and then spit out ``$0011$.'' Something like this:

\begin{diagram}

  \node at (-6, 1.25) {\textsf{ALU}};
  \node at (-6, 0.75) {\small{Adds ``$0001$'' and ``$0010$''}};
  \draw (-8, 0) -- (-4, 0) -- (-4, 2) -- (-8, 2) -- (-8, 0);
  \node at (-3.375, 1.75) {\textsf{O}};
  \draw (-4, 1.75) -- (-3.75, 1.75);
  \draw (-3.75, 2) -- (-3, 2) -- (-3, 1.5) -- (-3.75, 1.5) -- (-3.75, 2);
  \draw[color=gray]
    (-2.8, 2) -- (-2.8, 1.5) -- (-2.7, 1.5) -- (-2.7, 2) -- (-2.8, 2);
  \draw (-3, 2) -- (-2.5, 2) -- (-2.5, 1.5) -- (-3, 1.5);
  \node at (-3.375, 1) {\textsf{N}};
  \draw (-4, 1) -- (-3.75, 1);
  \draw (-3.75, 1.25) -- (-3, 1.25) -- (-3, 0.75) -- (-3.75, 0.75) -- (-3.75, 1.25);
  \draw[color=gray]
    (-2.8, 1.25) -- (-2.8, 0.75) -- (-2.7, 0.75) -- (-2.7, 1.25) -- (-2.8, 1.25);
  \draw (-3, 1.25) -- (-2.5, 1.25) -- (-2.5, 0.75) -- (-3, 0.75);
  \node at (-3.375, 0.25) {\textsf{Z}};
  \draw (-4, 0.25) -- (-3.75, 0.25);
  \draw (-3.75, 0.5) -- (-3, 0.5) -- (-3, 0) -- (-3.75, 0) -- (-3.75, 0.5);
  \draw[color=gray]
    (-2.8, 0.5) -- (-2.8, 0) -- (-2.7, 0) -- (-2.7, 0.5) -- (-2.8, 0.5);
  \draw (-3, 0.5) -- (-2.5, 0.5) -- (-2.5, 0) -- (-3, 0);

  \node at (-8, 4.5) {feed in};
  \node at (-8, 4) {``$0001$''};
  \node at (-8, 3.5) {as one input};
  \draw[->] (-8, 3) -- (-7, 2.25);

  \node at (-4, 4.5) {feed in};
  \node at (-4, 4) {``$0010$'' as};
  \node at (-4, 3.5) {another input};
  \draw[->] (-4, 3) -- (-5, 2.25);  
  
  \draw[->] (-6, -0.25) -- (-6, -1);
  \node at (-6, -1.5) {outputs};
  \node at (-6, -2) {``$0011$''};

\end{diagram}

Let's talk about those little 1-bit registers now. The ``\textsf{Z}'' register is called the \vocab{Zero Flag}. When the ALU performs a computation, if the result is a zero, it will turn on that ``$\textsf{Z}$'' bit. For instance, suppose we ask the ALU to add zero and zero. The result of zero plus zero is zero. So then, the ALU will spit out ``$0000$'' as its output, but it will also turn on its zero flag. So, it will look like this:

\begin{diagram}

  \node at (-6, 1) {\textsf{ALU}};
  \draw (-8, 0) -- (-4, 0) -- (-4, 2) -- (-8, 2) -- (-8, 0);
  \node at (-3.375, 1.75) {\textsf{O}};
  \draw (-4, 1.75) -- (-3.75, 1.75);
  \draw (-3.75, 2) -- (-3, 2) -- (-3, 1.5) -- (-3.75, 1.5) -- (-3.75, 2);
  \draw[color=gray]
    (-2.8, 2) -- (-2.8, 1.5) -- (-2.7, 1.5) -- (-2.7, 2) -- (-2.8, 2);
  \draw (-3, 2) -- (-2.5, 2) -- (-2.5, 1.5) -- (-3, 1.5);
  \node at (-3.375, 1) {\textsf{N}};
  \draw (-4, 1) -- (-3.75, 1);
  \draw (-3.75, 1.25) -- (-3, 1.25) -- (-3, 0.75) -- (-3.75, 0.75) -- (-3.75, 1.25);
  \draw[color=gray]
    (-2.8, 1.25) -- (-2.8, 0.75) -- (-2.7, 0.75) -- (-2.7, 1.25) -- (-2.8, 1.25);
  \draw (-3, 1.25) -- (-2.5, 1.25) -- (-2.5, 0.75) -- (-3, 0.75);
  \node at (-3.375, 0.25) {\textsf{Z}};
  \draw (-4, 0.25) -- (-3.75, 0.25);
  \draw (-3.75, 0.5) -- (-3, 0.5) -- (-3, 0) -- (-3.75, 0) -- (-3.75, 0.5);
  \draw[color=gray,fill=black]
    (-2.8, 0.5) -- (-2.8, 0) -- (-2.7, 0) -- (-2.7, 0.5) -- (-2.8, 0.5);
  \draw (-3, 0.5) -- (-2.5, 0.5) -- (-2.5, 0) -- (-3, 0);

  \node at (-8, 4.5) {feed in};
  \node at (-8, 4) {``$0000$''};
  \node at (-8, 3.5) {as one input};
  \draw[->] (-8, 3) -- (-7, 2.25);

  \node at (-4, 4.5) {feed in};
  \node at (-4, 4) {``$0000$'' as};
  \node at (-4, 3.5) {another input};
  \draw[->] (-4, 3) -- (-5, 2.25);  
  
  \draw[->] (-6, -0.25) -- (-6, -1);
  \node at (-6, -1.5) {outputs};
  \node at (-6, -2) {``$0000$''};

\end{diagram}

What about the ``\textsf{N}'' register? The ALU will turn this one on if the result is a negative number. We haven't talked about how to represent negative numbers inside the machine, but it can be done, and if the result of the ALU's computation is negative, it will turn on this ``\textsf{N}'' bit. We call the ``\textsf{N}'' register the \vocab{Negative Flag}.

What about the ``\textsf{O}'' flag? Suppose we try to add $1111$ and $0010$. What is the result? Well, if the ALU can only handle 4-bit numbers, it can't produce a number bigger than $1111$. So, it will wrap around back to zero (like a clock), and start adding again from there. If the ALU wraps around like this, we call it an \vocab{overflow}. If an overflow happens during a computation, the ALU will turn on the ``\textsf{O}'' bit. Hence, we call the ``\textsf{O}'' register the \vocab{Overflow Flag}.


%%%%%%%%%%%%%%%%%%%%%%%%%%%%%%%%%%%%%%%%%
%%%%%%%%%%%%%%%%%%%%%%%%%%%%%%%%%%%%%%%%%
\section{The CPU}

\newthought{At this point}, we have all the basic pieces we need to put together a \vocab{CPU} (short for the ``Central Processing Unit,'' or just the \vocab{processor}). A processor contains some scratch registers and an ALU, so it can perform computations. It also has one more register, called an \vocab{Instruction Pointer}, or ``IP'' for short. This register holds an address to memory, but we'll talk more about what this does in the next chapter. For now, it is enough to just put together a picture of a CPU. Here it is:

\begin{diagram}

  % The CPU

  \draw (-8.1, -1.1) -- (-8.1, 4.1) -- (-1.65, 4.1) -- (-1.65, -1.1) -- (-8.1, -1.1);
  \draw[fill=black]
    (-5.8, -1.1) -- (-4.2, -1.1) -- (-4.2, -1.7) -- (-5.8, -1.7) -- (-5.8, -1.1);
  \node[color=white] at (-5, -1.4) {\textsf{CPU}};

  \node at (-7.5, 3.75) {\textsf{R0}};
  \draw (-7, 3.5) -- (-8, 3.5) -- (-8, 4) -- (-7, 4);
  \draw[color=gray]
    (-6.7, 3.5) -- (-6.7, 4) -- (-6.8, 4) -- (-6.8, 3.5) -- (-6.7, 3.5);
  \draw[color=gray]
    (-6.2, 3.5) -- (-6.2, 4) -- (-6.3, 4) -- (-6.3, 3.5) -- (-6.2, 3.5);
  \draw[color=gray]
    (-5.7, 3.5) -- (-5.7, 4) -- (-5.8, 4) -- (-5.8, 3.5) -- (-5.7, 3.5);
  \draw[color=gray]
    (-5.2, 3.5) -- (-5.2, 4) -- (-5.3, 4) -- (-5.3, 3.5) -- (-5.2, 3.5);
  \draw (-7, 3.5) -- (-6.5, 3.5) -- (-6.5, 4) -- (-7, 4) -- (-7, 3.5);
  \draw (-6.5, 3.5) -- (-6, 3.5) -- (-6, 4) -- (-6.5, 4);
  \draw (-6, 3.5) -- (-5.5, 3.5) -- (-5.5, 4) -- (-6, 4);
  \draw (-5.5, 3.5) -- (-5, 3.5) -- (-5, 4) -- (-5.5, 4);

  \node at (-4.25, 3.75) {\textsf{R1}};
  \draw (-3.75, 3.5) -- (-4.75, 3.5) -- (-4.75, 4) -- (-3.75, 4);
  \draw[color=gray]
    (-3.45, 3.5) -- (-3.45, 4) -- (-3.55, 4) -- (-3.55, 3.5) -- (-3.45, 3.5);
  \draw[color=gray]
    (-2.95, 3.5) -- (-2.95, 4) -- (-3.05, 4) -- (-3.05, 3.5) -- (-2.95, 3.5);
  \draw[color=gray]
    (-2.45, 3.5) -- (-2.45, 4) -- (-2.55, 4) -- (-2.55, 3.5) -- (-2.45, 3.5);
  \draw[color=gray]
    (-1.95, 3.5) -- (-1.95, 4) -- (-2.05, 4) -- (-2.05, 3.5) -- (-1.95, 3.5);
  \draw (-3.75, 3.5) -- (-3.25, 3.5) -- (-3.25, 4) -- (-3.75, 4) -- (-3.75, 3.5);
  \draw (-3.25, 3.5) -- (-2.75, 3.5) -- (-2.75, 4) -- (-3.25, 4);
  \draw (-2.75, 3.5) -- (-2.25, 3.5) -- (-2.25, 4) -- (-2.75, 4);
  \draw (-2.25, 3.5) -- (-1.75, 3.5) -- (-1.75, 4) -- (-2.25, 4);

  \node at (-7.5, 2.75) {\textsf{R2}};
  \draw (-7, 2.5) -- (-8, 2.5) -- (-8, 3) -- (-7, 3);
  \draw[color=gray]
    (-6.7, 2.5) -- (-6.7, 3) -- (-6.8, 3) -- (-6.8, 2.5) -- (-6.7, 2.5);
  \draw[color=gray]
    (-6.2, 2.5) -- (-6.2, 3) -- (-6.3, 3) -- (-6.3, 2.5) -- (-6.2, 2.5);
  \draw[color=gray]
    (-5.7, 2.5) -- (-5.7, 3) -- (-5.8, 3) -- (-5.8, 2.5) -- (-5.7, 2.5);
  \draw[color=gray]
    (-5.2, 2.5) -- (-5.2, 3) -- (-5.3, 3) -- (-5.3, 2.5) -- (-5.2, 2.5);
  \draw (-7, 2.5) -- (-6.5, 2.5) -- (-6.5, 3) -- (-7, 3) -- (-7, 2.5);
  \draw (-6.5, 2.5) -- (-6, 2.5) -- (-6, 3) -- (-6.5, 3);
  \draw (-6, 2.5) -- (-5.5, 2.5) -- (-5.5, 3) -- (-6, 3);
  \draw (-5.5, 2.5) -- (-5, 2.5) -- (-5, 3) -- (-5.5, 3);

  \node at (-4.25, 2.75) {\textsf{R3}};
  \draw (-3.75, 2.5) -- (-4.75, 2.5) -- (-4.75, 3) -- (-3.75, 3);
  \draw[color=gray]
    (-3.45, 2.5) -- (-3.45, 3) -- (-3.55, 3) -- (-3.55, 2.5) -- (-3.45, 2.5);
  \draw[color=gray]
    (-2.95, 2.5) -- (-2.95, 3) -- (-3.05, 3) -- (-3.05, 2.5) -- (-2.95, 2.5);
  \draw[color=gray]
    (-2.45, 2.5) -- (-2.45, 3) -- (-2.55, 3) -- (-2.55, 2.5) -- (-2.45, 2.5);
  \draw[color=gray]
    (-1.95, 2.5) -- (-1.95, 3) -- (-2.05, 3) -- (-2.05, 2.5) -- (-1.95, 2.5);
  \draw (-3.75, 2.5) -- (-3.25, 2.5) -- (-3.25, 3) -- (-3.75, 3) -- (-3.75, 2.5);
  \draw (-3.25, 2.5) -- (-2.75, 2.5) -- (-2.75, 3) -- (-3.25, 3);
  \draw (-2.75, 2.5) -- (-2.25, 2.5) -- (-2.25, 3) -- (-2.75, 3);
  \draw (-2.25, 2.5) -- (-1.75, 2.5) -- (-1.75, 3) -- (-2.25, 3);

  \node at (-6, 1) {\textsf{ALU}};
  \draw (-8, 0) -- (-4, 0) -- (-4, 2) -- (-8, 2) -- (-8, 0);
  \node at (-3.375, 1.75) {\textsf{O}};
  \draw (-4, 1.75) -- (-3.75, 1.75);
  \draw (-3.75, 2) -- (-3, 2) -- (-3, 1.5) -- (-3.75, 1.5) -- (-3.75, 2);
  \draw[color=gray]
    (-2.8, 2) -- (-2.8, 1.5) -- (-2.7, 1.5) -- (-2.7, 2) -- (-2.8, 2);
  \draw (-3, 2) -- (-2.5, 2) -- (-2.5, 1.5) -- (-3, 1.5);
  \node at (-3.375, 1) {\textsf{N}};
  \draw (-4, 1) -- (-3.75, 1);
  \draw (-3.75, 1.25) -- (-3, 1.25) -- (-3, 0.75) -- (-3.75, 0.75) -- (-3.75, 1.25);
  \draw[color=gray]
    (-2.8, 1.25) -- (-2.8, 0.75) -- (-2.7, 0.75) -- (-2.7, 1.25) -- (-2.8, 1.25);
  \draw (-3, 1.25) -- (-2.5, 1.25) -- (-2.5, 0.75) -- (-3, 0.75);
  \node at (-3.375, 0.25) {\textsf{Z}};
  \draw (-4, 0.25) -- (-3.75, 0.25);
  \draw (-3.75, 0.5) -- (-3, 0.5) -- (-3, 0) -- (-3.75, 0) -- (-3.75, 0.5);
  \draw[color=gray]
    (-2.8, 0.5) -- (-2.8, 0) -- (-2.7, 0) -- (-2.7, 0.5) -- (-2.8, 0.5);
  \draw (-3, 0.5) -- (-2.5, 0.5) -- (-2.5, 0) -- (-3, 0);

  \node at (-7.5, -0.75) {\textsf{IP}};
  \draw (-7, -0.5) -- (-8, -0.5) -- (-8, -1) -- (-7, -1);
  \draw[color=gray]
    (-6.7, -0.5) -- (-6.7, -1) -- (-6.8, -1) -- (-6.8, -0.5) -- (-6.7, -0.5);
  \draw[color=gray]
    (-6.2, -0.5) -- (-6.2, -1) -- (-6.3, -1) -- (-6.3, -0.5) -- (-6.2, -0.5);
  \draw[color=gray]
    (-5.7, -0.5) -- (-5.7, -1) -- (-5.8, -1) -- (-5.8, -0.5) -- (-5.7, -0.5);
  \draw[color=gray]
    (-5.2, -0.5) -- (-5.2, -1) -- (-5.3, -1) -- (-5.3, -0.5) -- (-5.2, -0.5);
  \draw (-7, -0.5) -- (-6.5, -0.5) -- (-6.5, -1) -- (-7, -1) -- (-7, -0.5);
  \draw (-6.5, -0.5) -- (-6, -0.5) -- (-6, -1) -- (-6.5, -1);
  \draw (-6, -0.5) -- (-5.5, -0.5) -- (-5.5, -1) -- (-6, -1);
  \draw (-5.5, -0.5) -- (-5, -0.5) -- (-5, -1) -- (-5.5, -1);

\end{diagram}

You can see the four scratch registers $R0$, $R1$, $R2$, and $R3$; you can see the ALU with its overflow flag, negative flag, and zero flag; and you can see the instruction pointer (called $IP$). All together, that makes a CPU.

The final step in building a computer is to attach the memory (our RAM). So, the CPU and the memory together make up the computer's internals. Like this:

\begin{diagram}

  % The Memory

  \draw (-1.6, -0.1) -- (-1.6, 4.1) -- (4.1, 4.1) -- (4.1, -0.1) -- (-1.6, -0.1);
  \draw[fill=black]
    (0.25, -0.1) -- (2.75, -0.1) -- (2.75, -0.7) -- (0.25, -0.7) -- (0.25, -0.1);
  \node[color=white] at (1.5, -0.4) {\textsf{Memory}};

  \node at (-0.75, 3.75) {\textsf{0000}};
  \draw (0, 3.5) -- (-1.5, 3.5) -- (-1.5, 4) -- (0, 4);
  \draw[color=gray] (0.2, 3.5) -- (0.2, 4) -- (0.3, 4) -- (0.3, 3.5) -- (0.2, 3.5);
  \draw[color=gray] (0.7, 3.5) -- (0.7, 4) -- (0.8, 4) -- (0.8, 3.5) -- (0.7, 3.5);
  \draw[color=gray] (1.2, 3.5) -- (1.2, 4) -- (1.3, 4) -- (1.3, 3.5) -- (1.2, 3.5);
  \draw[color=gray] (1.7, 3.5) -- (1.7, 4) -- (1.8, 4) -- (1.8, 3.5) -- (1.7, 3.5);
  \draw[color=gray] (2.2, 3.5) -- (2.2, 4) -- (2.3, 4) -- (2.3, 3.5) -- (2.2, 3.5);
  \draw[color=gray] (2.7, 3.5) -- (2.7, 4) -- (2.8, 4) -- (2.8, 3.5) -- (2.7, 3.5);
  \draw[color=gray] (3.2, 3.5) -- (3.2, 4) -- (3.3, 4) -- (3.3, 3.5) -- (3.2, 3.5);
  \draw[color=gray] (3.7, 3.5) -- (3.7, 4) -- (3.8, 4) -- (3.8, 3.5) -- (3.7, 3.5);
  \draw (0, 3.5) -- (0.5, 3.5) -- (0.5, 4) -- (0, 4) -- (0, 3.5);
  \draw (0.5, 3.5) -- (1, 3.5) -- (1, 4) -- (0.5, 4);
  \draw (1, 3.5) -- (1.5, 3.5) -- (1.5, 4) -- (1, 4);
  \draw (1.5, 3.5) -- (2, 3.5) -- (2, 4) -- (1.5, 4);
  \draw (2, 3.5) -- (2.5, 3.5) -- (2.5, 4) -- (2, 4);
  \draw (2.5, 3.5) -- (3, 3.5) -- (3, 4) -- (2.5, 4);
  \draw (3, 3.5) -- (3.5, 3.5) -- (3.5, 4) -- (3, 4);
  \draw (3.5, 3.5) -- (4, 3.5) -- (4, 4) -- (3.5, 4);
  
  \node at (-0.75, 3.25) {\textsf{0001}};
  \draw (0, 3) -- (-1.5, 3) -- (-1.5, 3.5) -- (0, 3.5);
  \draw[color=gray] (0.2, 3) -- (0.2, 3.5) -- (0.3, 3.5) -- (0.3, 3) -- (0.2, 3);
  \draw[color=gray] (0.7, 3) -- (0.7, 3.5) -- (0.8, 3.5) -- (0.8, 3) -- (0.7, 3);
  \draw[color=gray] (1.2, 3) -- (1.2, 3.5) -- (1.3, 3.5) -- (1.3, 3) -- (1.2, 3);
  \draw[color=gray] (1.7, 3) -- (1.7, 3.5) -- (1.8, 3.5) -- (1.8, 3) -- (1.7, 3);
  \draw[color=gray] (2.2, 3) -- (2.2, 3.5) -- (2.3, 3.5) -- (2.3, 3) -- (2.2, 3);
  \draw[color=gray] (2.7, 3) -- (2.7, 3.5) -- (2.8, 3.5) -- (2.8, 3) -- (2.7, 3);
  \draw[color=gray] (3.2, 3) -- (3.2, 3.5) -- (3.3, 3.5) -- (3.3, 3) -- (3.2, 3);
  \draw[color=gray] (3.7, 3) -- (3.7, 3.5) -- (3.8, 3.5) -- (3.8, 3) -- (3.7, 3);
  \draw (0, 3) -- (0.5, 3) -- (0.5, 3.5) -- (0, 3.5) -- (0, 3);
  \draw (0.5, 3) -- (1, 3) -- (1, 3.5) -- (0.5, 3.5);
  \draw (1, 3) -- (1.5, 3) -- (1.5, 3.5) -- (1, 3.5);
  \draw (1.5, 3) -- (2, 3) -- (2, 3.5) -- (1.5, 3.5);
  \draw (2, 3) -- (2.5, 3) -- (2.5, 3.5) -- (2, 3.5);
  \draw (2.5, 3) -- (3, 3) -- (3, 3.5) -- (2.5, 3.5);
  \draw (3, 3) -- (3.5, 3) -- (3.5, 3.5) -- (3, 3.5);
  \draw (3.5, 3) -- (4, 3) -- (4, 3.5) -- (3.5, 3.5);

  \node at (-0.75, 2.75) {\textsf{0010}};
  \draw (0, 2.5) -- (-1.5, 2.5) -- (-1.5, 3) -- (0, 3);
  \draw[color=gray] (0.2, 2.5) -- (0.2, 3) -- (0.3, 3) -- (0.3, 2.5) -- (0.2, 2.5);
  \draw[color=gray] (0.7, 2.5) -- (0.7, 3) -- (0.8, 3) -- (0.8, 2.5) -- (0.7, 2.5);
  \draw[color=gray] (1.2, 2.5) -- (1.2, 3) -- (1.3, 3) -- (1.3, 2.5) -- (1.2, 2.5);
  \draw[color=gray] (1.7, 2.5) -- (1.7, 3) -- (1.8, 3) -- (1.8, 2.5) -- (1.7, 2.5);
  \draw[color=gray] (2.2, 2.5) -- (2.2, 3) -- (2.3, 3) -- (2.3, 2.5) -- (2.2, 2.5);
  \draw[color=gray] (2.7, 2.5) -- (2.7, 3) -- (2.8, 3) -- (2.8, 2.5) -- (2.7, 2.5);
  \draw[color=gray] (3.2, 2.5) -- (3.2, 3) -- (3.3, 3) -- (3.3, 2.5) -- (3.2, 2.5);
  \draw[color=gray] (3.7, 2.5) -- (3.7, 3) -- (3.8, 3) -- (3.8, 2.5) -- (3.7, 2.5);
  \draw (0, 2.5) -- (0.5, 2.5) -- (0.5, 3) -- (0, 3) -- (0, 2.5);
  \draw (0.5, 2.5) -- (1, 2.5) -- (1, 3) -- (0.5, 3);
  \draw (1, 2.5) -- (1.5, 2.5) -- (1.5, 3) -- (1, 3);
  \draw (1.5, 2.5) -- (2, 2.5) -- (2, 3) -- (1.5, 3);
  \draw (2, 2.5) -- (2.5, 2.5) -- (2.5, 3) -- (2, 3);
  \draw (2.5, 2.5) -- (3, 2.5) -- (3, 3) -- (2.5, 3);
  \draw (3, 2.5) -- (3.5, 2.5) -- (3.5, 3) -- (3, 3);
  \draw (3.5, 2.5) -- (4, 2.5) -- (4, 3) -- (3.5, 3);
  
  \node at (-0.75, 2.25) {\textsf{0011}};
  \draw (0, 2) -- (-1.5, 2) -- (-1.5, 2.5) -- (0, 2.5);
  \draw[color=gray] (0.2, 2) -- (0.2, 2.5) -- (0.3, 2.5) -- (0.3, 2) -- (0.2, 2);
  \draw[color=gray] (0.7, 2) -- (0.7, 2.5) -- (0.8, 2.5) -- (0.8, 2) -- (0.7, 2);
  \draw[color=gray] (1.2, 2) -- (1.2, 2.5) -- (1.3, 2.5) -- (1.3, 2) -- (1.2, 2);
  \draw[color=gray] (1.7, 2) -- (1.7, 2.5) -- (1.8, 2.5) -- (1.8, 2) -- (1.7, 2);
  \draw[color=gray] (2.2, 2) -- (2.2, 2.5) -- (2.3, 2.5) -- (2.3, 2) -- (2.2, 2);
  \draw[color=gray] (2.7, 2) -- (2.7, 2.5) -- (2.8, 2.5) -- (2.8, 2) -- (2.7, 2);
  \draw[color=gray] (3.2, 2) -- (3.2, 2.5) -- (3.3, 2.5) -- (3.3, 2) -- (3.2, 2);
  \draw[color=gray] (3.7, 2) -- (3.7, 2.5) -- (3.8, 2.5) -- (3.8, 2) -- (3.7, 2);
  \draw (0, 2) -- (0.5, 2) -- (0.5, 2.5) -- (0, 2.5) -- (0, 2);
  \draw (0.5, 2) -- (1, 2) -- (1, 2.5) -- (0.5, 2.5);
  \draw (1, 2) -- (1.5, 2) -- (1.5, 2.5) -- (1, 2.5);
  \draw (1.5, 2) -- (2, 2) -- (2, 2.5) -- (1.5, 2.5);
  \draw (2, 2) -- (2.5, 2) -- (2.5, 2.5) -- (2, 2.5);
  \draw (2.5, 2) -- (3, 2) -- (3, 2.5) -- (2.5, 2.5);
  \draw (3, 2) -- (3.5, 2) -- (3.5, 2.5) -- (3, 2.5);
  \draw (3.5, 2) -- (4, 2) -- (4, 2.5) -- (3.5, 2.5);

  \node at (-0.75, 1.75) {\textsf{0100}};
  \draw (0, 1.5) -- (-1.5, 1.5) -- (-1.5, 2) -- (0, 2);
  \draw[color=gray] (0.2, 1.5) -- (0.2, 2) -- (0.3, 2) -- (0.3, 1.5) -- (0.2, 1.5);
  \draw[color=gray] (0.7, 1.5) -- (0.7, 2) -- (0.8, 2) -- (0.8, 1.5) -- (0.7, 1.5);
  \draw[color=gray] (1.2, 1.5) -- (1.2, 2) -- (1.3, 2) -- (1.3, 1.5) -- (1.2, 1.5);
  \draw[color=gray] (1.7, 1.5) -- (1.7, 2) -- (1.8, 2) -- (1.8, 1.5) -- (1.7, 1.5);
  \draw[color=gray] (2.2, 1.5) -- (2.2, 2) -- (2.3, 2) -- (2.3, 1.5) -- (2.2, 1.5);
  \draw[color=gray] (2.7, 1.5) -- (2.7, 2) -- (2.8, 2) -- (2.8, 1.5) -- (2.7, 1.5);
  \draw[color=gray] (3.2, 1.5) -- (3.2, 2) -- (3.3, 2) -- (3.3, 1.5) -- (3.2, 1.5);
  \draw[color=gray] (3.7, 1.5) -- (3.7, 2) -- (3.8, 2) -- (3.8, 1.5) -- (3.7, 1.5);
  \draw (0, 1.5) -- (0.5, 1.5) -- (0.5, 2) -- (0, 2) -- (0, 1.5);
  \draw (0.5, 1.5) -- (1, 1.5) -- (1, 2) -- (0.5, 2);
  \draw (1, 1.5) -- (1.5, 1.5) -- (1.5, 2) -- (1, 2);
  \draw (1.5, 1.5) -- (2, 1.5) -- (2, 2) -- (1.5, 2);
  \draw (2, 1.5) -- (2.5, 1.5) -- (2.5, 2) -- (2, 2);
  \draw (2.5, 1.5) -- (3, 1.5) -- (3, 2) -- (2.5, 2);
  \draw (3, 1.5) -- (3.5, 1.5) -- (3.5, 2) -- (3, 2);
  \draw (3.5, 1.5) -- (4, 1.5) -- (4, 2) -- (3.5, 2);
  
  \node at (-0.75, 1.25) {\textsf{0101}};
  \draw (0, 1) -- (-1.5, 1) -- (-1.5, 1.5) -- (0, 1.5);
  \draw[color=gray] (0.2, 1) -- (0.2, 1.5) -- (0.3, 1.5) -- (0.3, 1) -- (0.2, 1);
  \draw[color=gray] (0.7, 1) -- (0.7, 1.5) -- (0.8, 1.5) -- (0.8, 1) -- (0.7, 1);
  \draw[color=gray] (1.2, 1) -- (1.2, 1.5) -- (1.3, 1.5) -- (1.3, 1) -- (1.2, 1);
  \draw[color=gray] (1.7, 1) -- (1.7, 1.5) -- (1.8, 1.5) -- (1.8, 1) -- (1.7, 1);
  \draw[color=gray] (2.2, 1) -- (2.2, 1.5) -- (2.3, 1.5) -- (2.3, 1) -- (2.2, 1);
  \draw[color=gray] (2.7, 1) -- (2.7, 1.5) -- (2.8, 1.5) -- (2.8, 1) -- (2.7, 1);
  \draw[color=gray] (3.2, 1) -- (3.2, 1.5) -- (3.3, 1.5) -- (3.3, 1) -- (3.2, 1);
  \draw[color=gray] (3.7, 1) -- (3.7, 1.5) -- (3.8, 1.5) -- (3.8, 1) -- (3.7, 1);
  \draw (0, 1) -- (0.5, 1) -- (0.5, 1.5) -- (0, 1.5) -- (0, 1);
  \draw (0.5, 1) -- (1, 1) -- (1, 1.5) -- (0.5, 1.5);
  \draw (1, 1) -- (1.5, 1) -- (1.5, 1.5) -- (1, 1.5);
  \draw (1.5, 1) -- (2, 1) -- (2, 1.5) -- (1.5, 1.5);
  \draw (2, 1) -- (2.5, 1) -- (2.5, 1.5) -- (2, 1.5);
  \draw (2.5, 1) -- (3, 1) -- (3, 1.5) -- (2.5, 1.5);
  \draw (3, 1) -- (3.5, 1) -- (3.5, 1.5) -- (3, 1.5);
  \draw (3.5, 1) -- (4, 1) -- (4, 1.5) -- (3.5, 1.5);

  \node at (-0.75, 0.75) {\textsf{0110}};
  \draw (0, 0.5) -- (-1.5, 0.5) -- (-1.5, 1) -- (0, 1);
  \draw[color=gray] (0.2, 0.5) -- (0.2, 1) -- (0.3, 1) -- (0.3, 0.5) -- (0.2, 0.5);
  \draw[color=gray] (0.7, 0.5) -- (0.7, 1) -- (0.8, 1) -- (0.8, 0.5) -- (0.7, 0.5);
  \draw[color=gray] (1.2, 0.5) -- (1.2, 1) -- (1.3, 1) -- (1.3, 0.5) -- (1.2, 0.5);
  \draw[color=gray] (1.7, 0.5) -- (1.7, 1) -- (1.8, 1) -- (1.8, 0.5) -- (1.7, 0.5);
  \draw[color=gray] (2.2, 0.5) -- (2.2, 1) -- (2.3, 1) -- (2.3, 0.5) -- (2.2, 0.5);
  \draw[color=gray] (2.7, 0.5) -- (2.7, 1) -- (2.8, 1) -- (2.8, 0.5) -- (2.7, 0.5);
  \draw[color=gray] (3.2, 0.5) -- (3.2, 1) -- (3.3, 1) -- (3.3, 0.5) -- (3.2, 0.5);
  \draw[color=gray] (3.7, 0.5) -- (3.7, 1) -- (3.8, 1) -- (3.8, 0.5) -- (3.7, 0.5);
  \draw (0, 0.5) -- (0.5, 0.5) -- (0.5, 1) -- (0, 1) -- (0, 0.5);
  \draw (0.5, 0.5) -- (1, 0.5) -- (1, 1) -- (0.5, 1);
  \draw (1, 0.5) -- (1.5, 0.5) -- (1.5, 1) -- (1, 1);
  \draw (1.5, 0.5) -- (2, 0.5) -- (2, 1) -- (1.5, 1);
  \draw (2, 0.5) -- (2.5, 0.5) -- (2.5, 1) -- (2, 1);
  \draw (2.5, 0.5) -- (3, 0.5) -- (3, 1) -- (2.5, 1);
  \draw (3, 0.5) -- (3.5, 0.5) -- (3.5, 1) -- (3, 1);
  \draw (3.5, 0.5) -- (4, 0.5) -- (4, 1) -- (3.5, 1);
  
  \node at (-0.75, 0.25) {\textsf{0111}};
  \draw (0, 0) -- (-1.5, 0) -- (-1.5, 0.5) -- (0, 0.5);
  \draw[color=gray] (0.2, 0) -- (0.2, 0.5) -- (0.3, 0.5) -- (0.3, 0) -- (0.2, 0);
  \draw[color=gray] (0.7, 0) -- (0.7, 0.5) -- (0.8, 0.5) -- (0.8, 0) -- (0.7, 0);
  \draw[color=gray] (1.2, 0) -- (1.2, 0.5) -- (1.3, 0.5) -- (1.3, 0) -- (1.2, 0);
  \draw[color=gray] (1.7, 0) -- (1.7, 0.5) -- (1.8, 0.5) -- (1.8, 0) -- (1.7, 0);
  \draw[color=gray] (2.2, 0) -- (2.2, 0.5) -- (2.3, 0.5) -- (2.3, 0) -- (2.2, 0);
  \draw[color=gray] (2.7, 0) -- (2.7, 0.5) -- (2.8, 0.5) -- (2.8, 0) -- (2.7, 0);
  \draw[color=gray] (3.2, 0) -- (3.2, 0.5) -- (3.3, 0.5) -- (3.3, 0) -- (3.2, 0);
  \draw[color=gray] (3.7, 0) -- (3.7, 0.5) -- (3.8, 0.5) -- (3.8, 0) -- (3.7, 0);
  \draw (0, 0) -- (0.5, 0) -- (0.5, 0.5) -- (0, 0.5) -- (0, 0);
  \draw (0.5, 0) -- (1, 0) -- (1, 0.5) -- (0.5, 0.5);
  \draw (1, 0) -- (1.5, 0) -- (1.5, 0.5) -- (1, 0.5);
  \draw (1.5, 0) -- (2, 0) -- (2, 0.5) -- (1.5, 0.5);
  \draw (2, 0) -- (2.5, 0) -- (2.5, 0.5) -- (2, 0.5);
  \draw (2.5, 0) -- (3, 0) -- (3, 0.5) -- (2.5, 0.5);
  \draw (3, 0) -- (3.5, 0) -- (3.5, 0.5) -- (3, 0.5);
  \draw (3.5, 0) -- (4, 0) -- (4, 0.5) -- (3.5, 0.5);

  % The CPU

  \draw (-8.1, -1.1) -- (-8.1, 4.1) -- (-1.65, 4.1) -- (-1.65, -1.1) -- (-8.1, -1.1);
  \draw[fill=black]
    (-5.8, -1.1) -- (-4.2, -1.1) -- (-4.2, -1.7) -- (-5.8, -1.7) -- (-5.8, -1.1);
  \node[color=white] at (-5, -1.4) {\textsf{CPU}};

  \node at (-7.5, 3.75) {\textsf{R0}};
  \draw (-7, 3.5) -- (-8, 3.5) -- (-8, 4) -- (-7, 4);
  \draw[color=gray]
    (-6.7, 3.5) -- (-6.7, 4) -- (-6.8, 4) -- (-6.8, 3.5) -- (-6.7, 3.5);
  \draw[color=gray]
    (-6.2, 3.5) -- (-6.2, 4) -- (-6.3, 4) -- (-6.3, 3.5) -- (-6.2, 3.5);
  \draw[color=gray]
    (-5.7, 3.5) -- (-5.7, 4) -- (-5.8, 4) -- (-5.8, 3.5) -- (-5.7, 3.5);
  \draw[color=gray]
    (-5.2, 3.5) -- (-5.2, 4) -- (-5.3, 4) -- (-5.3, 3.5) -- (-5.2, 3.5);
  \draw (-7, 3.5) -- (-6.5, 3.5) -- (-6.5, 4) -- (-7, 4) -- (-7, 3.5);
  \draw (-6.5, 3.5) -- (-6, 3.5) -- (-6, 4) -- (-6.5, 4);
  \draw (-6, 3.5) -- (-5.5, 3.5) -- (-5.5, 4) -- (-6, 4);
  \draw (-5.5, 3.5) -- (-5, 3.5) -- (-5, 4) -- (-5.5, 4);

  \node at (-4.25, 3.75) {\textsf{R1}};
  \draw (-3.75, 3.5) -- (-4.75, 3.5) -- (-4.75, 4) -- (-3.75, 4);
  \draw[color=gray]
    (-3.45, 3.5) -- (-3.45, 4) -- (-3.55, 4) -- (-3.55, 3.5) -- (-3.45, 3.5);
  \draw[color=gray]
    (-2.95, 3.5) -- (-2.95, 4) -- (-3.05, 4) -- (-3.05, 3.5) -- (-2.95, 3.5);
  \draw[color=gray]
    (-2.45, 3.5) -- (-2.45, 4) -- (-2.55, 4) -- (-2.55, 3.5) -- (-2.45, 3.5);
  \draw[color=gray]
    (-1.95, 3.5) -- (-1.95, 4) -- (-2.05, 4) -- (-2.05, 3.5) -- (-1.95, 3.5);
  \draw (-3.75, 3.5) -- (-3.25, 3.5) -- (-3.25, 4) -- (-3.75, 4) -- (-3.75, 3.5);
  \draw (-3.25, 3.5) -- (-2.75, 3.5) -- (-2.75, 4) -- (-3.25, 4);
  \draw (-2.75, 3.5) -- (-2.25, 3.5) -- (-2.25, 4) -- (-2.75, 4);
  \draw (-2.25, 3.5) -- (-1.75, 3.5) -- (-1.75, 4) -- (-2.25, 4);

  \node at (-7.5, 2.75) {\textsf{R2}};
  \draw (-7, 2.5) -- (-8, 2.5) -- (-8, 3) -- (-7, 3);
  \draw[color=gray]
    (-6.7, 2.5) -- (-6.7, 3) -- (-6.8, 3) -- (-6.8, 2.5) -- (-6.7, 2.5);
  \draw[color=gray]
    (-6.2, 2.5) -- (-6.2, 3) -- (-6.3, 3) -- (-6.3, 2.5) -- (-6.2, 2.5);
  \draw[color=gray]
    (-5.7, 2.5) -- (-5.7, 3) -- (-5.8, 3) -- (-5.8, 2.5) -- (-5.7, 2.5);
  \draw[color=gray]
    (-5.2, 2.5) -- (-5.2, 3) -- (-5.3, 3) -- (-5.3, 2.5) -- (-5.2, 2.5);
  \draw (-7, 2.5) -- (-6.5, 2.5) -- (-6.5, 3) -- (-7, 3) -- (-7, 2.5);
  \draw (-6.5, 2.5) -- (-6, 2.5) -- (-6, 3) -- (-6.5, 3);
  \draw (-6, 2.5) -- (-5.5, 2.5) -- (-5.5, 3) -- (-6, 3);
  \draw (-5.5, 2.5) -- (-5, 2.5) -- (-5, 3) -- (-5.5, 3);

  \node at (-4.25, 2.75) {\textsf{R3}};
  \draw (-3.75, 2.5) -- (-4.75, 2.5) -- (-4.75, 3) -- (-3.75, 3);
  \draw[color=gray]
    (-3.45, 2.5) -- (-3.45, 3) -- (-3.55, 3) -- (-3.55, 2.5) -- (-3.45, 2.5);
  \draw[color=gray]
    (-2.95, 2.5) -- (-2.95, 3) -- (-3.05, 3) -- (-3.05, 2.5) -- (-2.95, 2.5);
  \draw[color=gray]
    (-2.45, 2.5) -- (-2.45, 3) -- (-2.55, 3) -- (-2.55, 2.5) -- (-2.45, 2.5);
  \draw[color=gray]
    (-1.95, 2.5) -- (-1.95, 3) -- (-2.05, 3) -- (-2.05, 2.5) -- (-1.95, 2.5);
  \draw (-3.75, 2.5) -- (-3.25, 2.5) -- (-3.25, 3) -- (-3.75, 3) -- (-3.75, 2.5);
  \draw (-3.25, 2.5) -- (-2.75, 2.5) -- (-2.75, 3) -- (-3.25, 3);
  \draw (-2.75, 2.5) -- (-2.25, 2.5) -- (-2.25, 3) -- (-2.75, 3);
  \draw (-2.25, 2.5) -- (-1.75, 2.5) -- (-1.75, 3) -- (-2.25, 3);

  \node at (-6, 1) {\textsf{ALU}};
  \draw (-8, 0) -- (-4, 0) -- (-4, 2) -- (-8, 2) -- (-8, 0);
  \node at (-3.375, 1.75) {\textsf{O}};
  \draw (-4, 1.75) -- (-3.75, 1.75);
  \draw (-3.75, 2) -- (-3, 2) -- (-3, 1.5) -- (-3.75, 1.5) -- (-3.75, 2);
  \draw[color=gray]
    (-2.8, 2) -- (-2.8, 1.5) -- (-2.7, 1.5) -- (-2.7, 2) -- (-2.8, 2);
  \draw (-3, 2) -- (-2.5, 2) -- (-2.5, 1.5) -- (-3, 1.5);
  \node at (-3.375, 1) {\textsf{N}};
  \draw (-4, 1) -- (-3.75, 1);
  \draw (-3.75, 1.25) -- (-3, 1.25) -- (-3, 0.75) -- (-3.75, 0.75) -- (-3.75, 1.25);
  \draw[color=gray]
    (-2.8, 1.25) -- (-2.8, 0.75) -- (-2.7, 0.75) -- (-2.7, 1.25) -- (-2.8, 1.25);
  \draw (-3, 1.25) -- (-2.5, 1.25) -- (-2.5, 0.75) -- (-3, 0.75);
  \node at (-3.375, 0.25) {\textsf{Z}};
  \draw (-4, 0.25) -- (-3.75, 0.25);
  \draw (-3.75, 0.5) -- (-3, 0.5) -- (-3, 0) -- (-3.75, 0) -- (-3.75, 0.5);
  \draw[color=gray]
    (-2.8, 0.5) -- (-2.8, 0) -- (-2.7, 0) -- (-2.7, 0.5) -- (-2.8, 0.5);
  \draw (-3, 0.5) -- (-2.5, 0.5) -- (-2.5, 0) -- (-3, 0);

  \node at (-7.5, -0.75) {\textsf{IP}};
  \draw (-7, -0.5) -- (-8, -0.5) -- (-8, -1) -- (-7, -1);
  \draw[color=gray]
    (-6.7, -0.5) -- (-6.7, -1) -- (-6.8, -1) -- (-6.8, -0.5) -- (-6.7, -0.5);
  \draw[color=gray]
    (-6.2, -0.5) -- (-6.2, -1) -- (-6.3, -1) -- (-6.3, -0.5) -- (-6.2, -0.5);
  \draw[color=gray]
    (-5.7, -0.5) -- (-5.7, -1) -- (-5.8, -1) -- (-5.8, -0.5) -- (-5.7, -0.5);
  \draw[color=gray]
    (-5.2, -0.5) -- (-5.2, -1) -- (-5.3, -1) -- (-5.3, -0.5) -- (-5.2, -0.5);
  \draw (-7, -0.5) -- (-6.5, -0.5) -- (-6.5, -1) -- (-7, -1) -- (-7, -0.5);
  \draw (-6.5, -0.5) -- (-6, -0.5) -- (-6, -1) -- (-6.5, -1);
  \draw (-6, -0.5) -- (-5.5, -0.5) -- (-5.5, -1) -- (-6, -1);
  \draw (-5.5, -0.5) -- (-5, -0.5) -- (-5, -1) -- (-5.5, -1);

\end{diagram}


%%%%%%%%%%%%%%%%%%%%%%%%%%%%%%%%%%%%%%%%%
%%%%%%%%%%%%%%%%%%%%%%%%%%%%%%%%%%%%%%%%%
\section{Summary}

\newthought{In this chapter}, we looked at how to represent numbers inside a computer, we learned about scratch registers, memory (RAM), the Arithmetic and Logic Unit (the ALU), and the Central Processing Unit (the CPU).

\end{document}
