\documentclass[../../../main.tex]{subfiles}
\begin{document}

%%%%%%%%%%%%%%%%%%%%%%%%%%%%%%%%%%%%%%%%%
%%%%%%%%%%%%%%%%%%%%%%%%%%%%%%%%%%%%%%%%%
%%%%%%%%%%%%%%%%%%%%%%%%%%%%%%%%%%%%%%%%%
\chapter{Further Reading}

To pursue algebras further, the following list may offer some helpful starting points. Much of the literature on algebra is focused on \vocab{groups}, but there are more complex types of algebraic structures called \vocab{rings} and \vocab{fields}, which also receive a lot of treatment. \vocab{Linear algebra} is another big sub-discipline of algebra. After getting comfortable with groups, one can easily turn their sites on linear algebra, rings, and fields.

For more introductory level discussions that explain concepts rather than theorems and proofs, the following may be helpful:

\begin{itemize}

  \item \citet[chs.~5--7]{Sawyer2007} offers a beginner level discussion of basic arithmetic and algebraic structures.
  
  \item \citet[chs.~3--4, 7, 8, and 14]{Sawyer1982} provides simple explanations of many aspects of abstract algebra. See chapters 3--4 and 7 for the development of abstract algebraic structures generally, chapter 14 for groups, and chapter 8 for linear algebra.
  
  \item \citet{Sawyer2018} offers one of the best discussions of rings and fields I know of. The discussion is aimed at explaining the basic intuitions behind how these structures lead to the kinds of calculations many learn in ``algebra'' at school.  
  
  \item \citet[chs.~6, 7, and 15]{Stewart1995} offers excellent introductory discussion of groups (chapter 7), rings and fields (chapter 6), and linear algebra (chapter 15).

  \item \citet[chs.~13 and 10]{StewartAndTall2015} offers a somewhat introductory discussion of groups (in chapter 13) and fields (chapter 10).

\end{itemize}

For accessible but more thorough discussions, these may be helpful:

\begin{itemize}

  \item \citet{DosReisAndDosReis2017} is the best beginner's book for algebra I know of. It is thorough, and slow paced, and explains most things very well. In addition, the answer to every exercise is explained fully. To learn algebra, this is a great first place to start.
  
  \item \citet{Pinter2010} is an accessible introduction to algebraic structures. It can be quite useful as a companion to read alongside \citet{DosReisAndDosReis2017} and \citet{BaumslagAndChandler1968}. There is one problem with the text: the exercises are full of typos and errors. This is too bad, because the exercises are often very interesting. That being said, if you want to do the exercises, answers to almost all of them can be found online if you look hard enough, and that will help you straighten out any errors that you can't straighten out yourself.
  
  \item \citet{BaumslagAndChandler1968} is another great text to start with, or to use as a second text after \citet{DosReisAndDosReis2017}. The notation is quite different from what most other books use, but the authors do explain their notation, so this problem is easily overcome. The nice thing about this text (and all Schaum's Outlines books) is that all exercises have solutions.
  
  \item \citet{Saracino2008} is a good textbook on abstract algebra. If you can find solutions online, this is a great text to work through after covering \citet{DosReisAndDosReis2017} or \citet{BaumslagAndChandler1968}.

  \item \citet{Schmidt1966} is an old, out-of-date text, but in many ways it is simple and can be quite accessible.
  
  \item \citet{Warner1990} is an advanced text that covers a great many aspects of modern algebra.

\end{itemize}

\end{document}
