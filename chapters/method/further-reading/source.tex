\documentclass[../../../main.tex]{subfiles}
\begin{document}

%%%%%%%%%%%%%%%%%%%%%%%%%%%%%%%%%%%%%%%%%
%%%%%%%%%%%%%%%%%%%%%%%%%%%%%%%%%%%%%%%%%
%%%%%%%%%%%%%%%%%%%%%%%%%%%%%%%%%%%%%%%%%
\chapter{Further Reading}

To pursue matters of the mathematical method further, the following list may offer some helpful starting points. 

When reading and studying any topic in \math/, it can sometimes be best to work through multiple texts in parallel. Work on one until you get stuck, then go to the other. Then back to the first, and so on. And always try to do the exercises.

\begin{itemize}

  \item \citet[chs.~1--2]{Wilder2012} provides an overview of the mathematical method (as well as many other interesting topics relevant to the mathematical approach).

  \item \citet{StewartAndTall2015} offers a good introduction to \mathical/ logic too.

  \item \citet{Devlin2012} provides a simple overview of basic \mathical/ logic.

  \item \citet[]{BlackburnAndBos2005} gives a simple introduction to logic from the perspective of computational linguistics. If the \mathical/ way of presenting logic feels impenetrable, try this one and the next.

  \item \citet[chs. 2--3]{DowtyEtAl1981} offers another but more thorough introduction to logic from the perspective of formal linguistics. This might also be helpful if the \mathical/ way of presenting logic feels too curt.
  
  \item \citet{Teller1989} is one of the best introductions to modern formal logic that I know of. If there is one text to learn logic from, I would recommend this one. If you can find it in print for an affordable price, great. There is nothing better than having the text at hand, to underline, highlight, scratch notes in the margins, and so on. It is also freely available from the author, at https://tellerprimer.ucdavis.edu/

\end{itemize}

\noindent
For further introductions to modern \mathical/ topics as a whole, see the following, which cover many of the topics we cover in this book, and more:

\begin{itemize}

  \item \citet{Steinhart2018} provides an accessible introduction to a variety of \mathical/ topics.

  \item \citet{BurgerAndStarbird2010} is an extremely accessible introduction to many \mathical/ topics.
  
  \item \citet{Stewart1995} is an old classic, but it does discuss a variety of \mathical/ topics in an accessible manner.

\end{itemize}

\noindent
For detailed guidance on writing \mathical/ proofs:

\begin{itemize}

  \item \citet{Velleman2019} provides an excellent tutorial on the business of proper proof writing. Doing all the exercises in this book is well worth it.
  
  \item \citet{Wolf1998} is another fairly accessible guide to the business of proper proof writing.

  \item \citet[chs.~7--8]{MaddenAndAubrey2017} provides a detailed discussion of logic and proof writing that can be tackled after working through \citet{Velleman2019}.

\end{itemize}




\end{document}
