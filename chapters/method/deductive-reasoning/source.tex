\documentclass[../../../main.tex]{subfiles}
\begin{document}

%%%%%%%%%%%%%%%%%%%%%%%%%%%%%%%%%%%%%%%%%
%%%%%%%%%%%%%%%%%%%%%%%%%%%%%%%%%%%%%%%%%
%%%%%%%%%%%%%%%%%%%%%%%%%%%%%%%%%%%%%%%%%
\chapter{Deductive Reasoning}

\newtopic{P}{ure \math/ is a discipline} done through pure thinking. In this respect, \math/ is a lot like philosophy. Both philosophers and \mathers/ can do the entirety of their career's work, while sitting in an armchair and just \emph{thinking}. 

\begin{ponder}
  What do \emph{you} think \math/ is? What do you think the \vocab{proper subject} of study is for a \mather/? Where do you think you got those ideas? 
\end{ponder}

Of course, a \mather/ might have some scratch paper beside them, to scribble things down while they think (and a big blackboard is really just a big piece of scratch paper). And naturally, a \mather/ might want to write down and publish the conclusions they come to. But in principle, the real work consists in the thinking part.

So what exactly is it that \mathers/ think about? If you are anything like me, you may have grown up with the impression that \math/ is all about \emph{calculation}. Perhaps you got this idea from school (I did). For a long time, I think I thought that the job of a \mather/ was to sit there, and calculate bigger and bigger numbers. (And I guess if you manage to calculate the biggest number so far, you win the game.)

\begin{ponder}
  How would you define a \vocab{number}? Can you give me a definition of a number without using the word ``number'' (or an equivalent concept) in your definition?
\end{ponder}

But \math/ is not really like this. Not at all. Of course, there can be some calculation involved, on occasion, when it is helpful. But calculation is really a part of \emph{applied \math/}. When it comes to \emph{pure \math/}, which is the topic we are concerned with here, the \mather/ is typically not interested merely in calculating some particular number. On the contrary, the \mather/ is usually much more interested in figuring out facts about a whole \emph{class} of numbers, or figuring out what a number even \emph{is}.

\begin{terminology}
An \vocab{abstract structure} is a structure that is lifted away from irrelevant concrete details. For instance, instead of looking at a system of roads that connect cities, we can look at it as points, connected by lines.
\end{terminology}

Moreover, a great deal of \math/ doesn't even deal with numbers. More fundamentally, \mathers/ spend their time dealing with \emph{abstract structures} of various kinds. So, in what follows, let us to think about \math/ as the \vocab{abstract study of structures}. The \mather/ is interested in understanding in a deep way what these structures are like, and what kinds of properties they have.


%%%%%%%%%%%%%%%%%%%%%%%%%%%%%%%%%%%%%%%%%
%%%%%%%%%%%%%%%%%%%%%%%%%%%%%%%%%%%%%%%%%
\section{What is Deductive Reasoning?}

\newthought{How do we study abstract structures}? How do we learn things about abstract structures? \Mathers/ employ a very distinctive method to come up with the conclusions that they come up with. This method is sometimes referred to as \emph{deductive reasoning}.

Essentially, it is a logic game. Before we start the game, we agree to accept some basic \emph{starting points}. Then, during the game, our goal is to derive new conclusions --- conclusions that follow logically from our starting points. The primary rule of the game is this: you cannot use anything but cold, hard logic to figure out the new conclusions. 

\begin{aside}
  \begin{remark}
    Think of the \vocab{starting points} as a small number of statements that we all agree to accept as ``given'' during the course of our investigation. Why do we need this? Well, we need to start somewhere, so in order to get our investigations going, we agree on some particular starting points.
  \end{remark}
\end{aside}

During play, we are \emph{strict} about the rules. We insist that we cannot use anything other than the information contained in our agreed upon starting points, and pure logic to move forward. So no new information is permitted, once we have started. Hence, the \emph{only} things we can conclude, are things that \vocab{strictly follow} (logically) from our starting points.

That is what deductive reasoning is all about. \vocab{Deductive reasoning} is a process of reasoning where we start from some agreed upon starting points, and we use nothing but pure logic to figure what follows. We call it ``deductive'' because we use rigorous logical thinking to \emph{deduce} the conclusions from our given starting points. 

You might be thinking of Sherlock Holmes at this point, and you would be right to do so. Sherlock Holmes is one of fiction's most famous practitioners of deductive reasoning. He would always start with a set of facts gathered as evidence (these are his starting points), and invariably he would then use pure logic to figure out further facts that follow logically from his given starting points. This is the heart and soul of deductive reasoning.


%%%%%%%%%%%%%%%%%%%%%%%%%%%%%%%%%%%%%%%%%
%%%%%%%%%%%%%%%%%%%%%%%%%%%%%%%%%%%%%%%%%
\section{An Example}

\newthought{To get a sense} of how the deductive process works, let's consider a made-up example, which is not taken from the realm of \math/. Suppose we live in a small town which has a quaint public park at its center called ``Hamfordton Park.'' This nice little park has the usual amenities. It has a playground for the children, it has a large grass field for playing sports, it has a few sidewalks meandering around the perimeter, and there is a selection of benches resting under shady trees.

Suppose that after an unfortunate accident, the city council established a statute which reads as follows:

\begin{quote}
  \highlight{Statute 2.34}: \\
  Vehicles are prohibited in Hamfordton Park.
\end{quote} 

\noindent
For our purposes here, let's reword this a little bit, so that it has the shape of an ``if-then'' statement, like this:

\begin{aside}
  \begin{remark}
    It is okay to reword statements, so long as we \vocab{preserve the meaning} of the original statement. English (or any other natural language) can be quite vague, so it is often quite useful to try and reword things to make matters as explicit and exact as we can.
  \end{remark}
\end{aside}

\begin{quote}
  \highlight{Statute 2.34$^{*}$} (Statute 2.34 reworded): \\
  \emph{If} a vehicle is in Hamfordton Park, \emph{then} it is in Hamfordton park illegally.
\end{quote}

\noindent
Suppose next that our local sheriff reports that a motorcycle is in the park. Our question now is this: is the motorcycle in the park illegally? Before reading the next paragraph, pause for a moment, and think about it. What do you think? Is the motorcycle in the park illegally, or not? And \emph{why} did you come up with the answer you came up with? Or, did you conclude that we can't answer the question with the information we have. If so, why not?

Well, let's look at the statute (to be exact, let's look at our reworded Statute 2.34$^{*}$). According to the statute, if a vehicle is in the park, it's there illegally. But what exactly counts as a \vocab{vehicle}? The statute does not say, so we cannot answer this.

To proceed, let's agree on a definition of vehicles. Let us say that, for legal purposes, we will define a vehicle like this:

\begin{aside}
  \begin{remark}
    Think about \vocab{Definition A}. What sorts of things count as vehicles, \emph{according to this definition}? And what sorts of things do \emph{not} count as vehicles? Can you come up with some examples? Try to come up with some outlandish examples too. A good \mather/ will always think about limit cases!
  \end{remark}
\end{aside}

\begin{quote}
    \highlight{Definition A} (Vehicle): \\
  A vehicle is anything that has four wheels.
\end{quote}

\noindent
Okay, so now there are two statements that we have agreed to: we have agreed to \vocab{Statute 2.34$^{*}$}, and \vocab{Definition A}. These are our starting points.

Let's return to our question about the motorcycle. Is it in the park illegally? It should be clear now that the answer is no: the motorcycle is \emph{not} in Hamfordton Park illegally. Why? Because \emph{Statute 2.34$^{*}$} and \emph{Definition A} together say that only four-wheeled things are prohibited, and a motorcycle has two wheels, so it cannot count as a ``vehicle'' in this situation, \emph{according to the statute and definition that we agreed upon}.

\begin{aside}
  \begin{example}
    If we strictly follow \emph{Statute 2.34$^{*}$} and \emph{Definition A}, is a wheelchair prohibited in the park? A matchbox car? A bicycle?
  \end{example}
\end{aside}

Here is another case. Suppose the sheriff reports that an 18-wheeler is parked in the middle of Hamfordton Park. Is it there illegally? The answer is no, it is not parked there illegally, \emph{according to the definition of vehicle} that we agreed to. An 18-wheeler has, well, 18 wheels, rather than four wheels, so it too does not count as a ``vehicle'' in this situation.

It may seem counterintuitive to think that an 18-wheeler is not a vehicle. Maybe you think it is also counterintuitive to think that a motorcycle is not a vehicle. But that is not the point.

\begin{aside}
  \begin{remark}
    If you don't think motorcycles or 18-wheelers should be in Hamfordton Park, you might petition the city council to change the legal definition of a ``vehicle.'' Can you think of a definition that would capture the right cases?
  \end{remark}
\end{aside}

The point of this exercise is to notice the way that we drew conclusions. We stuck to our agreed upon statute and definition strictly, in order to see what follows logically from them. We concluded \emph{only} what followed strictly from our agreed upon statute and definition.

This kind of reasoning is very much like what happens in \math/. In \math/, we always start with some agreed upon starting points, and then we do all our reasoning from there. And we are \vocab{strict}, in that we only let ourselves use pure logic to derive conclusions from our starting points. That is the essence of \vocab{deductive reasoning}.


%%%%%%%%%%%%%%%%%%%%%%%%%%%%%%%%%%%%%%%%%
%%%%%%%%%%%%%%%%%%%%%%%%%%%%%%%%%%%%%%%%%
\section{Mathematical Style}

\newthought{In a mathematical investigation}, coming up with the conclusions (or even coming up with the right starting points) can sometimes be easy, and sometimes difficult. Sometimes you can see a few conclusions immediately, but other times it can take a lot of creative thinking and time to come up with the most interesting conclusions. There is no mechanical way to do it! It typically takes lots of scratch paper, and lots of patience to sit with the ideas and chew on them, until things become clear.

When a \mather/ has finished their work and come up with the conclusions they care about, then they will sit down and write up the results of their investigation. Over the centuries, \mathers/ have settled into a particular way of writing, so they can set out their starting points, their reasoning, and their conclusions, in a compact way. There is specific terminology they use to refer to these things too. A \mather/ will usually write down their results in the following order:

\begin{terminology}
  A \vocab{definition} stipulates precisely what the meaning of some term or concept is intended to be. An \vocab{axiom} is any other statement we should accept as given for the investigation.
\end{terminology}

\begin{terminology}
  A \vocab{theorem} (or \vocab{lemma}) is a conclusion that we have come up with, using nothing but our agreed upon starting points and pure logic.
\end{terminology}

\begin{terminology}
  A \vocab{proof} is a written description of the logical steps that lead from the given starting points to the theorem (or lemma).
\end{terminology}

\begin{enumerate}

  \item \highlight{Starting points}. A \mather/ will first write down their starting points. Two kinds of starting points are common: definitions and axioms. A \vocab{definition} stipulates what the \mather/ intends the meaning of some term or idea to be for the course of their investigation. An \vocab{axiom} is any statement that the \mather/ wants us to accept as given for the course of their investigation
          
  \item \highlight{Conclusions}. Next, a \mather/ will write down the conclusions that they came up with. \Mathers/ call their conclusions \vocab{theorems}. Sometimes, they call them \vocab{lemmas}. But really a lemma is just a theorem too.

  \item \highlight{The reasoning}. After each conclusion (theorem or lemma) is stated, a \mather/ will write down a short description of the reasoning one should use to derive that conclusion from the given starting points. This written description is called a \vocab{proof}. Anybody who wants to follow along in the investigation should be able to read the proof description, and follow along with the logic, to see how to get to the conclusion too. 

\end{enumerate}

\noindent
So, to sum up, \mathers/ will first state their definitions and axioms, then they state their theorems, one after another. After each theorem, they provide a proof, which is a short description of the reasoning that gets you from the starting points to the conclusion (the theorem).

\begin{terminology}
  \Mathers/ never accept a theorem without proof! Without proof, it is just a \vocab{conjecture}.
\end{terminology}

It is important to note that a \mather/ will never state a theorem or a lemma without immediately providing a proof for it. If a \mather/ suspects that some conclusion follows, but they cannot prove it, then they call it a \vocab{conjecture}. So a theorem or lemma is something proved, whereas a conjecture is something that we might suspect to be true, but we have not found a way to prove it, and so we cannot be sure that it is in fact true.




%%%%%%%%%%%%%%%%%%%%%%%%%%%%%%%%%%%%%%%%%
%%%%%%%%%%%%%%%%%%%%%%%%%%%%%%%%%%%%%%%%%
\section{Internalizing It}

\newthought{Reading \math/ is a very intellectual} kind of activity. It is probably not right to say that we \emph{read} \math/. It is perhaps better to say that we \emph{sit} with it, so to speak, and \emph{chew} on it.

\begin{aside}
  \begin{remark}
    It is probably worth noting that \mathers/ get very comfortable being in that emotional state of \emph{not understanding something right away}. Mathematical ideas are usually grasped and internalized very slowly.
  \end{remark}
\end{aside}

The basic goal when reading \math/ is to \vocab{internalize} the ideas. This takes time, and it requires an active thought process (not a \emph{passive} thought process). Definitions, axioms, theorems, and the like are very precise, and very carefully worded. We can't just read over them quickly. 

When we read and think about mathematical concepts, what we need to do is stop and read them carefully, break with them down, and work out their meaning and importance slowly. We often have to force our minds to get active in the process, by scribbling things down on scratch paper, and trying out examples on our own, to make sure we understand each and every part of the concept. 

It often happens that a \mather/ will sit with a single definition or theorem for hours, days, or even longer, as they work to understand it. This active process of understanding and internalizing takes time and patience, but the result can be incredibly rewarding.


%%%%%%%%%%%%%%%%%%%%%%%%%%%%%%%%%%%%%%%%%
%%%%%%%%%%%%%%%%%%%%%%%%%%%%%%%%%%%%%%%%%
\section{Summary}

In this chapter, we learned that \mathers/ use \emph{deductive reasoning} to do their work. 

\begin{itemize}

  \item \vocab{Deductive reasoning} is a process of reasoning where we first agree to some starting points (namely, \vocab{definitions} and \vocab{axioms}), and then we use nothing but our starting points and \vocab{pure logic} to draw further conclusions.
  
  \item During our reasoning, we are \vocab{strict} in that we do not allow any new information to seep into our considerations. The only thing we can use as we do our thinking is our agreed upon starting points (our accepted definitions and axioms), and pure logic.

  \item The conclusions that we draw using this deductive process are called \vocab{theorems} (or sometimes \vocab{lemmas}). 
  
  \item For any given theorem (or lemma), \mathers/ \emph{always} write down a description of the logical reasoning that leads to that theorem (or lemma). This description is called a \vocab{proof}.
  
\end{itemize}

In this chapter, we did not spend much time talking about how exactly one is supposed to use ``pure logic'' to deduce further conclusions from our starting points. What does that actually look like? We also did not talk much about what theorems and proofs look like. In the next chapter, we will begin to look at just those things. 


\end{document}
